\documentclass[12pt,letterpaper]{report}
\usepackage[a4paper,left=2.5cm, right=2.5cm, top=2.5cm, bottom=2.5cm]{geometry}

\usepackage[pdftex]{graphicx}

\usepackage[romanian]{babel} % Suport limba romana
\usepackage{times}           % Times New Roman
\usepackage{setspace}        % Spatierea intre linii
\usepackage{titlesec}        % Capitole, sectiuni, subsectiuni..
\usepackage{enumerate}       % Liste
\usepackage{amsmath}         % Formatari in ecuatii
\usepackage{amsfonts}        % Formatari speciale
\usepackage{amsthm}          % Modulul de teoreme
\usepackage{hyperref}        % Linkuri in document
\usepackage{algorithm2e}     % Pseudocod inline
\usepackage{tikz}            % Creare de diagramare
\usepackage{float}           % Figuri in paralel cu text
\usepackage{url}             % Referinte URL

% Calea catre imagini
\graphicspath{{images/}}

% Tabel de continut
\setcounter{secnumdepth}{1}
\setcounter{tocdepth}{1}

% Formatarea capitolelor
\titleformat{\chapter}
  {\normalfont\LARGE\bfseries}{\thechapter}{1em}{}

% Spatierea intre capitole, sectiuni si subsectiuni
\titlespacing*{\chapter}{0pt}{0ex plus 1ex minus .2ex}{2.3ex plus .2ex}
\titlespacing*{\section} {0pt}{1ex plus 1ex minus .2ex}{1.5ex plus .2ex}
\titlespacing*{\subsection} {0pt}{2ex plus 1ex minus .2ex}{1ex plus .2ex}

% Spatierea in jurul figurilor
\setlength{\belowcaptionskip}{-15pt}

% Spatierea subsolului
\setlength{\skip\footins}{0.5cm}

% Continut
\renewcommand{\contentsname}{Conținut}

% Teorema
\theoremstyle{definition} % Prefer stilul de la definitii si aici
  \newtheorem{thm}{Teorema}[chapter]

% Definitie
\theoremstyle{definition}
  \newtheorem{defn}[thm]{Definiția}

% Lema
\theoremstyle{definition}
  \newtheorem{lem}[thm]{Lema}

% Corolar
\theoremstyle{definition}
  \newtheorem{clr}[thm]{Corolar}

% Demonstratie
\renewcommand*{\proofname}{Demonstrație}

% Bibliografie
\renewcommand{\bibname}{Bibliografie}

% Capitole cu numere romane
\renewcommand \thechapter{\Roman{chapter}}

\begin{document}
  \doublespacing
  \begin{titlepage}
  \vspace{1 in}
  \begin{center}
    \hspace{4in} \includegraphics[width=0.3\textwidth]{unibuc}
    \\
    \huge
    \textbf{Algoritmi pentru cuplaje \\ in grafuri bipartite} \\
    \vspace{0.8 in}

    \Large
    \textbf{Ciobanu Bogdan} \\
    Lect.dr. Irofti Paul \\

    \vspace*{\fill}
    Iulie, 2020 \\
    Facultatea de Matematica si Informatica, Universitatea Bucuresti \\

  \end{center}
\end{titlepage}


  \thispagestyle{empty}
\begin{center}
    \Large
    \textbf{Algoritmi pentru cuplaje \\ în grafuri bipartite}

    \vspace{0.4cm}
    {Ciobanu Bogdan}

    \vspace{0.9cm}
    \textbf{Rezumat}

\end{center}
\onehalfspacing
\paragraph{}
În această lucrare ne propunem să prezentăm metode alternative pentru calcule
legate de cuplaje în grafuri bipartite. Plecăm de la algoritmi care folosesc
metode algebrice în detrimentul celor uzuale, bazate pe rețele de flux, și
facem o analiză asupra algoritmilor eficienți pentru operații de algebră
liniară pentru a înțelege cum îi putem folosi pentru a calcula cuplaje.
Folosim metode randomizate și demonstrăm probabilitațiile lor de succes.
Ne folosim de structura lor pentru a rezolva probleme care nu se pot rezolva
cu flux, iar pentru cele care se pot rezolva și cu flux, arătăm cum putem să
obținem complexități cel puțin la fel de bune, cu algoritmi care apoi se pot
paraleliza mai ușor, iar în practică sunt mai pretabili hardware-ului modern.

\begin{center}
    \Large
    \vspace{0.9cm}
    \textbf{Abstract}
\end{center}
\onehalfspacing
\paragraph{}
In this paper we will explain alternative methods for computing matchings in
bipartite graphs. We are starting from algorithms which use algebraic tehniques
as opposed to the usual choice, network flows, and we will examine how efficient
linear algebra algorithms can be used in matchings computation. We are using
randomized methods and we are proving their probability of success. We use their
structure to solve problems that could have never been solved with network flows,
but other problems as well, to show that we can achieve a theoretical time
complexity that is as good as their counterpart, but benefiting from the
efficiency of algebraic operations on modern hardware.


  % Numere italice pana la Introducere
  \pagestyle{plain}
  \pagenumbering{roman}
  \setcounter{page}{4}

  \tableofcontents
  \pagebreak

  % Numere arabice, pornind de la 1
  \pagestyle{plain}
  \pagenumbering{arabic}
  \setcounter{page}{1}
  \doublespacing

  \chapter{Introducere}

Cuplajul este o problemă fundamentală în teoria grafurilor. Aceasta are
aplicații vaste în informatica teoretică, fiind la baza unor probleme precum
izomorfismul de subarbori, acoperirea unui graf orientat aciclic cu număr minim
de cai disjuncte și detectarea obiectelor aflate în mișcare, dar și în chimie
pentru structura Kekule și indexul Hosoya. \par

De-alungul timpului s-au dezvoltat mulți algoritmi eficienți, din punct de
vedere al complexității de timp, pentru rezolvarea problemelor de cuplaj pe
diferite tipuri de grafuri. \par

În această lucrare ne vom concentra pe grafurile bipartite. În particular pentru
aceștia, cuplajul maximal, o problemă fundamentală de optimizare, se poate
calcula eficient folosind algoritmi de flux maxim. Acest lucru este totuși și o
piedică pentru arhitectură calculatoarelor moderne, fiind grei de paralelizat și
optimizat pentru accesul memoriei cache. De asemenea, pentru aceștia nu se
cunosc extensii în timp polinomial pentru aflarea altor rezultate teoretice
utile, cum ar fi numărul de cuplaje perfecte. Acest lucru a avut totuși să se
schimbe în 1995, când R. Motwani și P. Raghavan au arătat cum se poate folosi
matricea Edmonds, numită după Jack Edmonds care a avut multe contribuții în
domeniul optimizarilor, și algoritmi de algebră liniară pentru a afla mărimea
cuplajului maximal, în lucrarea ``Randomized algorithms''\cite{randomizedalgorithms}. \par

Algoritmii de algebră liniară au avantajul că admit implementări eficiente pe
procesoarele moderne, având un potențial mult mai mare pentru paralelism și
concurență. Observațiile pornind de la matricea lui Edmonds ne vor ajută să rezolvăm
probleme care nu au mai fost rezolvate până acum în timp polinomial. \par

Lucrarea este împărțită în două părți: contextul și problemele. Contextul este o
baza de informații folosite pentru a rezolve problemele. Acesta este compus în
mare parte de noțiuni de teoria grafurilor și algebră, cu accentul pus pe cea din
urmă. Apoi, prezentăm o colecție de problemele, unde fiecare a fost aleasă pentru
a ilustra:
\begin{itemize}
    \item fie cum algoritmii de cuplaj bazați pe algebră liniară o pot rezolva în
    timp polinomial, în timp ce algoritmii de flux nu pot îmbunătăți complexitatea
    exponențială.
    \item fie cum o problema care în aparență nu s-ar preta algoritmilor de cuplaj
    bazați pe algebră liniară, dar se poate rezolva cu flux, se poate de fapt rezolva
    și cu prima metodă având complexitate de timp cel puțin la fel de bună.
\end{itemize}

Problemele sunt alese fie din folclor, fie propuse la concursuri de
programare competitivă. Dintre acestea, problema ``Xor Matching'' a fost propusă
chiar de mine în 2018 și a fost inspirația pentru a scrie această lucrare.

  \chapter{Context}

\section{Concepte din algebra liniară}

\begin{defn}
  Un \textbf{monoid} $G = (A, \circ)$ este o structură algebrică formată dintr-o
  mulțime $A$ și o lege de compoziție asociativă și cu element neutru.
\end{defn}

\begin{defn}
  Un \textbf{grup} $G = (A, \circ)$ este o structură algebrică formată dintr-o
  mulțime $A$ și o lege de compoziție care respectă următoarele proprietăți:
  \begin{itemize}
      \item{$\forall\ x_{1}, x_{2} \in G, x_{1} \circ x_{2} \in G$.}
      \item{$\forall\ x_{1}, x_{2}, x_{3} \in G, (x_{1} \circ x_{2}) \circ x_{3} = x_{1} \circ (x_{2} \circ x_{3})$.}
      \item{$\exists\ e \in G \text{ astfel
      încât } \forall\ x \in G\ e \circ x = x \circ e = x$.}
      \item{$\forall\ x_{1} \in G\ \exists\ x_{2} \in G \ x_{1} \circ x_{2} = x_{2} \circ x_{1} = e$.}
  \end{itemize}
  În plus, acesta este abelian dacă
  $\forall\ x_{1}, x_{2} \in G\ x_{1} \circ x_{2} = x_{2} \circ x_{1}$.
\end{defn}

\begin{defn}
  Un \textbf{inel} $I = (A, +, *)$ este o structura algebrică formată dintr-o
  mulțime $A$ și două operații binare definite pe $A \times A$ cu valori tot în
  $A$, care respectă următoarele proprietăți:
  \begin{itemize}
      \item{$(A, +)$ este un grup abelian.}
      \item{$(A, *)$ este un monoid.}
      \item{Distributivitatea lui $*$ față de $+$.}
  \end{itemize}
\end{defn}

\begin{defn}
  Un \textbf{polinom} este o expresie construită dintr-una sau mai multe
  variabile și constante, folosind doar operații de adunare, scădere, înmulțire
  și ridicare la putere constanta pozitivă întreagă.
\end{defn}

\begin{defn}
  O \textbf{matrice} este un tabel dreptunghiular de elemente ale unui inel.
\end{defn}

\subsection{Interpolare Lagrange}

\textbf{Polinomul de interpolare Lagrange} este un polinom $P$ construit în așa
fel încât având $k + 1$ perechi
$(x_{1}, y_{1}), (x_{2}, y_{2}), \ldots, (x_{k+1}, y_{k+1})$ avem că
$P(x_{i}) = y_{i}\ \forall\ 1 \leq i \leq K + 1$. \par
Construcția lui $P$ a fost oferită de Joseph Louis Lagrange în $1795$:
\begin{equation}
  P(x) := \displaystyle\sum\limits_{i_{1}=0}^{k} (y_{i_{1}} * \prod_{1 \leq i_{2} \leq k + 1 \ \land \ i_{1} \neq i_{2}} \frac{x - x_{i_{2}}}{x_{i_{1}} - x_{i_{2}}})
\end{equation}

\section{Eliminare Gaussiană}
\label{gauss}

O metodă de a rezolva sisteme de ecuații liniare este \textbf{Eliminarea Gaussiană}.
Aceasta se mai numește și metoda \textbf{Gauss-Jordan} pentru că este o variație a
metodei lui Gauss, descrisă de către Jordan în anul 1887.

Algoritmul face eliminări secvențiale a tuturor variabilelor până când fiecare ecuație
rămâne cu o singură variabilă. Dacă o privim ca o ecuație $Ax = b$, atunci se poate
vedea că, în cazul în care $A$ este o matrice pătratică, atunci vrem să o transformăm
în matricea identitate și să rezolvăm sistemul pe acest caz trivial.

La primul pas, algoritmul împarte prima linie cu $a_{1, 1}$ și apoi adaugă o combinație
liniară a primei ecuații tuturor celorlalte astfel încât toți ceilalți coeficienți de pe
prima coloană să fie $0$; mai exact pentru rândul $a_{i}, i \geq 2$, vom adaugă primul rând
înmulțit cu $-a_{i}$. Toate aceste operații se execută și pe vectorul coloană $b$. Se
continuă aceleași operații pentru restul liniilor.

Este posibil ca acele valori să fie $0$. De aceea există mai multe euristici, în special gândite
pentru a stabiliza operațiile pe numere reale în virgulă flotantă. Euristica cea mai des folosită
în practică este să se caute valoarea cea mai mare absolută de pe acea coloană și să se interschimbe
cele două rânduri.

Complexitatea algoritmului este de $O(N^{2}M)$ dacă $A$ este de mărime $N \times M$, cu $N \leq M$.
În practică acest algoritm se poate aplica și atunci când se lucrează într-un corp finit.
Un caz des întâlnit este în $\mathbb{F}_{2}$. Pe modelul RAM acest caz se poate implementa chiar în
complexitate $O(\frac{N^{2}M}{w})$, unde $w$ este mărimea unui cuvânt, adică $64$ pentru procesoarele
moderne.

Metoda poate fi folosită și pentru a calcula inversa unei matrice. Tot ce este de făcut, pentru o matrice
$A$ de mărime $N \times N$ este să se execute algoritmul pe matricea $[A\ |\ I_{N}]$ de mărime $N \times 2N$,
iar matricea de pe blocul din dreapta (coloanele  cu indicemai mare ca $N$) va fi chiar $A^{-1}$.

Pentru calculul determinantului, putem executa algoritmul de înainte cu mențiunea că nu vom mai împărți fiecare
linie, dar vom înmulți acele valori de pivot și de fiecare dată când schimbăm două linii, schimbăm semnul
rezultatului. Dacă la un moment dat nu se găsește un pivot, atunci rezultatul este chiar $0$. Dacă se
dorește calculul rangului, atunci aceste este chiar numărul pivoților diferiți de $0$.

\pagebreak

\section{Teoria grafurilor}

\begin{defn}
  Un \textbf{graf neorientat} este o pereche ordonată $G(V, E)$, unde:
  \begin{itemize}
    \item{$V$ este mulțimea de noduri}.
    \item{$E \subseteq \{(x, y)\ |\ (x, y) \in V^2 \land x \neq y\}$ este
      mulțimea de muchii, care sunt perechi neordonate de noduri}.
  \end{itemize}
\end{defn}

\begin{defn}
  Un \textbf{arbore} este un graf neorientat conex aciclic.
\end{defn}

\begin{defn}
  Un \textbf{graf bipartit} este un graf neorientat $G(V, E)$ ale cărui noduri pot fi
  partiționate în două mulțimi disjuncte $V_{1}$ și $V_{2}$, altfel încât fiecare muchie
  conectează un nod din mulțimea $V_{1}$ cu un nod din mulțimea $V_{2}$. Atunci,
  se va nota cu $G(V_{1}, V_{2}, E)$.
\end{defn}

\begin{defn}
  Un \textbf{cuplaj} într-un graf este o mulțime de muchii fără noduri în comun.
\end{defn}

\begin{defn}
  \textbf{Matricea Edmonds} $A$ a grafului bipartit $G(U, V, E)$, unde
  $U = \{u_{1}, u_{2}, u_{3}, \ldots, u_{n}\}$,
  $V = \{v_{1}, v_{2}, v_{3}, \ldots, v_{n}\}$ și $x_{i,j}$ indeterminate, se definește ca:
   \begin{equation}
    A=
    \begin{cases}
      x_{i,j} & \text{dacă}\ (u_{i}, v_{j}) \in E \\
      0 & \text{altfel}
    \end{cases}
  \end{equation}
\end{defn}

\begin{thm}
  \label{edmonds}
  Un graf bipartit $G(U, V, E)$ admite cuplaj perfect dacă și numai dacă
  polinomul determinantului matricei sale Edmonds nu este polinomul nul.
\end{thm}

\begin{clr}
  Numărul de cuplaje perfecte este egal cu numărul monoamelor în polinomul $det(A_{i,j})$.
\end{clr}

\begin{clr}
  Rangul matricei Edmonds este egal cu mărimea cuplajului maximal.
\end{clr}

\pagebreak

\section{Lema Schwartz–Zippel}
\textbf{Lema Schwartz-Zippel} este o metodă utilă pentru a verifica într-un mod
probabilistic dacă un polinom în mai multe variabile este polinomul nul.

\begin{thm}
  Fie $P \in F \lbrack x_{1}, x_{2}, \ldots, x_{n} \rbrack$ un polinom nenul de grad $d \geq 0$
  peste corpul $F$ și $S$ o submultime finită a lui $F$. Alegem
  $r_{1}, r_{2}, \ldots, r_{n}$ uniform aleator și independent din $S$. Atunci:
  \begin{equation}
    \Pr \lbrack P(r_{1}, r_{2}, \ldots, r_{n}) = 0 \rbrack \leq \frac{d}{|S|}
  \end{equation}
\end{thm}

Acest rezultat are sens odată ce ne gândim pe cazul de baza, când $n=1$ deoarece
un polinom de grad $d$ are cel mult $d$ rădăcini. Întreagă demonstrație se
bazează pe inducție matematică și se poate consulta în \cite{schwartzzippel}.

\begin{thm}
  Lema Schwartz-Zippel se poate folosi în \ref{edmonds} pentru a verifica
  existența unui cuplaj perfect în timp polinomial.
\end{thm}

Fie $A$ \textbf{matricea Edmonds} asociată grafului. $\det(A)$ este un polinom
în $n^{2}$ variabile, de grad $n$. Alegem corpul pe care lucrăm că fiind
$\mathbb{F}_{p}$. Determinantul poate fi calculat sub $\mathbb{F}_{p}$ în timp $O(n^{3})$
folosind Eliminare Gaussiana. Chiar mai mult decât atât, acest calcul
poate fi făcut în timp $O(\log^{2} n)$ dacă avem la dispoziție $O(n^{3.5})$
procesoare \cite{paralleldet}. Probabilitatea de eșec a algoritmului este de $\frac{n}{p}$.

\pagebreak

\section{Lema de izolare}
\textbf{Lema de izolare} este o metodă de a reduce numărul de soluții ale unei
probleme la una singură, dacă aceasta există. Această se obține prin adăugarea
unor constrângeri aleatoare în așa fel încât, cu o probabilitate neglijabilă,
va există o singură soluție care satisface constragerile adiționale.

Lema a fost introdusă de Valiant și Vazirani în lucrarea ``NP is as easy as
detecting unique solutions'', publicată în 1986.

\begin{thm}
  \label{isolation}
  Fie $n$ și $m$ două numere întregi pozitive și $F$ o familie de submultimi a
  lui $\{1, 2, \ldots n\}$. Fie
  $w : \{1, 2, \ldots, n\} \to \{1, 2, \ldots, m\}$ o funcție care asociază
  valori uniform aleatoare și independente din codomeniu. Atunci, cu o
  probabilitate de cel puțin $1 - \frac{n}{m}$, va există o singură mulțime $S$ în
  $F$ pentru care $\displaystyle\sum\limits_{x\in S}{w(x)}$ este minimă.
\end{thm}

\pagebreak

\section{Algoritm paralelizabil pentru cuplaj}
\label{Algoritm paralelizabil pentru cuplaj}

În ``Matching is as easy as matrix inversion'' \cite{matchingezmatrix} se
descrie o metodă pentru a căuta cuplajul perfect folosind acest rezultat.
Considerăm $G(U, V, E)$ un graf bipartit. Asociem fiecărei muchii din $E$ un
cost uniform din mulțimea $\{1, 2, \ldots, 2|E|\}$. Conform \ref{isolation}, cu
probabilitate de cel puțin $\frac{1}{2}$ va exista un cuplaj unic de cost minim.
Considerăm $U = \{u_{1}, u_{2}, u_{3}, \ldots, u_{n}\}$,
$V = \{v_{1}, v_{2}, v_{3}, \ldots, v_{n}\}$ și costul aleator ales mai devreme muchiei
$(u_{i}, v_{j})$ că fiind $w_{i,j}$. Fie $A$ matricea Edmonds a acestui
graf, construită în felul următor:
\begin{equation}
A=
\begin{cases}
  2^{w_{i,j}} & \text{dacă}\ (u_{i}, v_{j}) \in E \\
  0 & \text{altfel}
\end{cases}
\end{equation}

Definim costul unui cuplaj ca fiind suma costurilor muchiilor alese în acea mulțime.
\begin{lem}
  Dacă $p$ este costul cuplajului minim și acesta este și unic, atunci $\det(A) \neq 0$, iar
  $2^{p} \ |\ \det(A)$ și $2^{p+1} \not| \ \det(A)$.
\end{lem}

Fie $\text{sgn} : S_{n} \to \{\pm 1\}$, $\text{sgn}(P)$ este $+1$ dacă $P$
este permutare pară, altfel $-1$. $\text{sgn}(P)$ se numește signatura
permutarii $P$.

\begin{equation}
  \label{isolationdet}
  \det(A) = \displaystyle\sum\limits_{P \in S_{n}} (\text{sgn}(P) * \prod_{1 \leq i \leq N} A_{i, P_{i}})
\end{equation}

Pentru fiecare cuplaj din $G$ putem asocia unic o permutare $P$ din $S_{n}$:
dacă muchia $(u_{i}, v_{j})$ este aleasă în cuplaj, atunci $P_{i} \equiv j$.
Pentru o permutare care nu are asociat un cuplaj perfect, aceasta nu va
contribui deloc la valoarea expresiei \ref{isolationdet}. \par

Dacă $P$ este permutarea asociată cuplajului de cost minim, atunci
$\prod_{1 \leq i \leq N} A_{i, P_{i}} = 2^{p}$. Valoarea celorlalte permutări
este fie $0$, fie o putere de $2$ mai mare decât $2^{p}$ (deoarece $P$ este
unică). Așadar, valoarea determinatului o să fie divizibilă cu $2^{p}$ și
mai mult, această va fi cel mai mare divizor de această formă.

\pagebreak

Pentru a determina mulțimea muchiilor din cuplaj, trebuie să mai facem câteva observații.

\begin{lem}
  Fie $A'$ matricea obținută eliminând linia $i$ și coloana $j$. Fie
  $C \def \det(A') * \frac{2^{w_{i,j}}}$. Muchia
  $(u_{i}, v_{j})$ face parte din cuplaj dacă și numai dacă $2^{p} \ | \ C$,
  $2^{p+1} \not| \ C$ și $C \neq 0$.
\end{lem}

În primul rând, valoarea $\det(A') * 2^{w_{i,j}}$ este chiar valoarea
determinantului calculat după formula de la \ref{isolationdet}, cu precizarea că
se vor considera doar permutările $P$ cu $P_{i} = j$. Acest lucru se datorează
faptului că permutările alese în calculul lui $\det(A')$ vor avea cu un element
mai puțin decât cele din $\det(A)$, iar pentru elementul lipsă înmulțim mereu cu
valorea asociată muchiei $(u_{i}, v_{j})$, adică $2^{w_{i,j}}$.
Aplicând apoi observațiile demonstrației pentru \ref{isolationdet} obținem
rezultatul dorit și în cele din urmă, un algoritm pentru determinarea cuplajului
perfect care nu se bazează pe rețele de flux, ci doar pe obiect de algebră
liniară: \par

\vspace{5 mm}

\begin{algorithm}[H]
 \label{cuplajizolare}
 \KwData{Graf bipartit G(U, V, E) care admite cuplaj perfect}
 \KwResult{Mulțimea muchiilor alese într-un posibil cuplaj perfect}
 \Begin{
  determină w\;
  construiește A\;
  $p \longleftarrow \det(A)$\;
  $C \longleftarrow \{\}$\;
  \For{$(u_{i}, v_{j}) \in E$}{
    determină $A'$\;
    \If{$\frac{\det(A') * 2^{w_{i,j}}}{2^{p}} \mod 2 = 1$}{
      $C \longleftarrow C \cup \{(u_{i}, v_{j})\}$\;
    }
  }
  \Return{$C$}\;
 }
 \caption{Cuplaj perfect lema de izolare}
\end{algorithm}

\pagebreak

În continuare vom presupune că putem efectua operațiile aritmetice de bază în timp $O(1)$.
Atunci algoritmul \ref{cuplajizolare} durează $O(n^{5})$, dacă folosim
algoritmul lui Gauss pentru calcularea determinantului. Putem să nu recalculăm
valoarea determinantului la fiecare iterație folosind:
\begin{equation}
  \det(A + uv^{T}) = (1 + v^{T} A^{-1} u) \det(A)
\end{equation}

Cu această proprietatea a determinantului, cunoscută și ca \textbf{Matrix
  determinant lemma} putem reduce complexitatea la $O(n^{4})$. În plus,
algoritmul se poate paraleliza bine, deoarece determinantul unei matrice
de mărime $n \times n$ se poate calcula în timp $O(\log^{2}n)$ pe $O(n^{3.5})$ procesoare.

Unicitatea soluției, garantata de \textbf{lema de izolare} permite procesoarelor
să execute și pasul următor, în care se testează dacă o muchie face parte din
cuplaj, în paralel, deoarece se îndreaptă către aceeași soluție, cu o
certitudine destul de mare. Așadar și această parte a algoritmului poate fi
calculată în timp polilogaritmic, dacă avem la dispoziție un număr polinomial de
procesoare.

În \cite{optimalmatching} se arată că acest algoritm poate fi rafinat la complexitate
$O(n^{3})$, observandu-se că o muchie $(u_{i}, v_{j})$ face parte din cuplaj dacă și
numai dacă $A^{-1}_{i, j} \neq 0$. Mai mult, folosindu-se de această proprietate și
de o schemă de a reduce eliminarea Gaussiana (\ref{gauss}) la $O(1)$ înmulțiri de matrici,
acest algoritm are complexitate $O(n^{w})$, unde $w < 2.38$, astfel obtinandu-se un
algoritm mai eficient decât cel bazat pe flux maxim, Hopcroft-Karp, care are complexitatea
$O(n^{2.5})$.

\pagebreak

\section{Cuplaj roșu-albastru}
\label{redbluematching}

Să considerăm un graf bipartit $G(U, V, E)$, unde $E_{1}$ este mulțimea
muchiilor roșii, $E_{2}$ este mulțimea muchiilor albastre, iar
$E = E_{1} \cup E_{2}$. Dorim să alfam dacă există un cuplaj perfect
în care se folosesc exact $k$ muchii roșii, știind că dacă acesta există, atunci
este și \textbf{unic}. \par

Aceasta problemă nu se poate rezolva cu algoritmi de flux maxim, spre deosebire
de celelalte probleme de până acum. Totuși, avem șanse să o rezolvam, dacă
procedam prin a modifica \textbf{matricea Edmonds} astfel încât să avem un mod
de a ne da seama de culoarea muchiilor alese în cuplaj:

\begin{equation}
  A=
  \begin{cases}
    y & \text{dacă}\ (u_{i}, v_{j}) \in E_{1} \\
    1 & \text{dacă}\ (u_{i}, v_{j}) \in E_{2} \\
    0 & \text{altfel}
  \end{cases}
\end{equation}

$\det(A)$ va fi un polinom în $y$ de grad $n$. $G$ va avea un cuplaj perfect cu
$k$ muchii roșii dacă coeficientul lui $y^{k}$ va fi nenul. Pentru a determina
acest număr putem să interpolăm polinomul $\det(A)$ folosind interpolarea
Lagrange. Tot ce rămâne de făcut este să evaluăm polinomul în $n+1$ puncte. \par

Pentru cazul general, când cuplajul roșu-albastru nu este neapărat unic, putem
folosi una din cele două leme să reducem problema la una mai simpla.

\subsection{Folosind Lema Schwartz-Zippel}

Pentru fiecare muchie $(u_{i}, v_{j})$ alegem o valoarea aleatoare $x_{i,j}$ dintr-o
mulțimea $S$. Construim matricea astfel:

\begin{equation}
  A=
  \begin{cases}
    y * x_{i,j} & \text{dacă}\ (u_{i}, v_{j}) \in E_{1} \\
    x_{i,j} & \text{dacă}\ (u_{i}, v_{j}) \in E_{2} \\
    0 & \text{altfel}
  \end{cases}
\end{equation}

Ca înainte, vom interpola $\det(A)$ pentru a afla coeficientul lui $y^{k}$.
Acest coeficient trebuie să fie nenul. Iar acum, cu probabilitate mare, acest
lucru este adevărat, pentru că dacă două polinoame se ``anulează'' unul pe
celalalt în urma calculului determinantului, atunci cu probabilitate de cel mult
$\frac{n}{|S|}$ acestea erau diferite.

\subsection{Folosind Lema de izolare}
Asemănător algoritmului prezentat la \ref{Algoritm paralelizabil pentru cuplaj}
și soluției precedente cu Lema Schwartz-Zippel, având valorile $w_{i,j}$ alese
din $\{1, 2, \ldots, 2|E_{1} \cup E_{2}|\}$, vom construi matricea astfel:

\begin{equation}
  A=
  \begin{cases}
    y * 2^{w_{i,j}} & \text{dacă}\ (u_{i}, v_{j}) \in E_{1} \\
    2^{w_{i,j}} & \text{dacă}\ (u_{i}, v_{j}) \in E_{2} \\
    0 & \text{altfel}
  \end{cases}
\end{equation}

În continuare rămâne să ne uităm la coeficientul lui $y^{k}$ în $\det(A)$. Spre
deosebire de lema \textbf{Schwartz-Zippel}, acesta variantă are posibilitatea să fie
paralelizabil, deoarece, cu probabilitate de cel puțin $\frac{1}{2}$, cuplajul
pe care vrem să-l găsim este acum unic.

\pagebreak

\section{Transformări Fourier}

\subsection{Transformarea Fourier Discreta \cite{fft}}
Considerăm $n = 2^{p}$ și următorul polinom de grad $n-1$:
\begin{equation}
  A = a_{0}x^{0} + a_{1}x^{1} + \ldots + a_{n-1}x^{n-1}
\end{equation}

Soluția ecuației $x^{n} = 1$ are $n$ soluții în mulțimea numerelor complexe.
Acestea sunt de forma $w_{p} = e^{\frac{2k\pi i}{n}}$ pentru $p$ un număr întreg
intre $0$ și $n-1$. Aceste rădăcini au multe proprietăți utile, iar cea pe care
urmează să o folosim este că $w_{p} = w_{0}^{p}$. \par
\textbf{Transformarea discretă Fourier} a polinomului $A$ este definită ca fiind
vectorul obținut prin evaluarea polinomului în rădăcinile unității:

\begin{equation}
  \label{dft}
  \text{DFT}(A) = (A(w_{0}), A(w_{1}), \ldots, A(w_{n-1})) \\
                = (A(w_{0}^{1}), A(w_{0}^{2}), \ldots, A(w_{0}^{n-1}))
\end{equation}

Similar, având vectorul din partea dreapta a ecuației \ref{dft}, \textbf{inversa
transformării discrete Fourieri} reconstituie coeficienții polinomului $A$:
$(a_{0}, a_{1}, \ldots a_{n-1})$. Vom nota această transformare cu
$\text{InverseDFT}$. \par

O aplicație utila este înmulțirea polinoamelor. Fie $\textbf{dot}$ operația de
înmulțire scalara a doi vectori. Aceasta se bazează pe o
proprietatea a transformării, și anume:
\begin{equation}
  \text{DFT}(A * B) = \text{dot}(\text{DFT}(A), \text{DFT}(B))
\end{equation}

Aceasta proprietate este utilă, fiind la baza unui algoritm rapid de înmulțire a
polinoamelor, deoarece $\text{DFT}(A)$ și $\text{InverseDFT}(A)$ se pot calcula
în timp $O(n \log n)$.

\subsection{Transformarea Hadamard}
\label{hadamard}

\textbf{Transformarea Hadamard} este o transformare Fourier generalizată,
executând o operație liniară pe $2^{n}$ numere reale. Aceasta este echivalenta
cu o transformare discretă $n$-dimensională Fourier de mărime
$2 \times 2 \ldots \times 2$. În mod intuitiv, se înmulțesc monoame a $n$
variabile, aplicandu-se modulo $x^{2}-1$ pe fiecare variabilă. Fie $p_{i_{1}, i_{2}} \in \{0, 1\}$:
\begin{equation}
  A = a_{0} x_{0}^{p_{0, 0}} x_{1}^{p_{0, 1}} \ldots x_{n-1}^{p_{0, n-1}}
  + a_{1} x_{0}^{p_{1, 0}} x_{1}^{p_{1, 1}} \ldots x_{n-1}^{p_{1, n-1}}
  + \ldots
  + a_{n-1} x_{0}^{p_{n-1, 0}} x_{1}^{p_{n-1, 1}} \ldots x_{n-1}^{p_{n-1, n-1}}
\end{equation}

Daca în cazul \textbf{transformării Fourieri discrete} obțineam
$C_{i+j} = \displaystyle\sum\limits A_{i} B_{j}$, în cazul \textbf{transformării Hadamard} obținem
$C_{i \oplus j} = \displaystyle\sum\limits A_{i} B_{j}$, unde $\oplus$ este operația de sau exclusiv
(xor) pe biți. Intuitiv, de ce se obține operație de xor este pentru că
cele $n$ monoame reprezentau biții fiecărui număr de la $0$ la $2^{n} - 1$.
Acestora la înmulțire li se aplica modulo $x^{2} - 1$, asemănător biților în
timpul unei operații de xor, care este ca o adunare, în urma căreia se aplica
modulo $2$, totul efectuat separat pe fiecare bit în parte.

\subsection{Algoritmul Fast Walsh-Hadamard}
\label{fasthadamard}

\textbf{Algoritmul Fast Walsh-Hadamard} \cite{fwht} este un algoritm eficient pentru a
calcula \textbf{transformarea Hadamard}. O implementare naivă a transformării
necesită complexitate de timp $O(2^{2n})$, în timp ce algoritmul pe care urmează
să-l prezentam necesita doar $O(n2^{n})$. Algoritmul se bazează pe paradigma
divide et impera, împărțind o problema de mărime $2^{n}$ în doua de mărime
$2^{n-1}$, urmărind definiția recursiva a matricei Hadamard:

\begin{equation}
  \label{hadamardmatrix}
  H_{n} =
  \begin{pmatrix}
    H_{n-1} & H_{n-1}\\
    H_{n-1} & -H_{n-1}
  \end{pmatrix}
\end{equation}

Ecuația \ref{hadamardmatrix} omite factorul de normalizare de
$\frac{1}{\sqrt{2}}$, care poate fi aplicat la final.

\begin{algorithm}[H]
  \DontPrintSemicolon
  \SetKwFunction{FMain}{FWHT}
  \SetKwProg{Fn}{Function}{:}{}
  \Fn{\FMain{$a$, $n$}}{
    \If{$n = -1$}{
      \KwRet\;
    }
    offset := $2^{n - 1}$\;
    \FMain($a[0 \ldots \text{offset}]$, n - 1)\;
    \FMain($a[\text{offset} \ldots 2^{n}]$, n - 1)\;
    \For{i in $0 \ldots 2^{n}$}{
      x := $a_{i}$\;
      y := $a_{i + \text{offset}}$\;
      $a_{i}$ := x + y\;
      $a_{i + \text{offset}}$ := x - y\;
    }
    \KwRet\;
  }
  \;
\end{algorithm}

\begin{algorithm}[H]
  \DontPrintSemicolon
  \SetKwFunction{FMain}{IFWHT}
  \SetKwProg{Fn}{Function}{:}{}
  \Fn{\FMain{$a$, $n$}}{
    \If{$n = -1$}{
      \KwRet\;
    }
    offset := $2^{n - 1}$\;
    \FMain($a[0 \ldots \text{offset}]$, n - 1)\;
    \FMain($a[\text{offset} \ldots 2^{n}]$, n - 1)\;
    \For{i in $0 \ldots 2^{n}$}{
      x := $a_{i}$\;
      y := $a_{i + \text{offset}}$\;
      $a_{i}$ := x - y\;
      $a_{i + \text{offset}}$ := x + y\;
    }
    \KwRet\;
  }
  \;
\end{algorithm}

\pagebreak

\section{Algoritmul lui Coppersmith}
\label{coppersmith}

În cadrul algoritmilor de algebră liniară pentru cuplaj avem des problema de a
calcula determinanți. De obicei, aceste grafuri nu vor fi atât de dense, deci
nici matricea Edmonds (\ref{edmonds}) nu va avea multe elemente nenule.

\textbf{Algoritmul lui Coppersmith} \cite{wiedemann} este un algoritm rapid pentru calcularea
vectorilor nucleu al unei matrice într-un corp finit. Acesta este în particular
eficient pentru matricele rare, cele care nu au atât de multe elemente nenule. \\

Considerăm $A$ o matrice pătratică de mărime $n \times n$ peste un corp finit $F$, atunci:

\begin{defn}
  Polinomul caracteristic al lui $A$ este definit că
  $p(\lambda) = \det(\lambda I_{n} - A)$.
\end{defn}

\begin{thm}
  \label{cayleyhamilton}
  \textbf{Cayley-Hamilton} spune că fiecare matrice pătratica într-un inel
  comutativ își satisface ecuația caracteristica, mai exact $p(A) = 0$.
\end{thm}

$A$ are un polinom monic $P$ peste corpul $F$ de grad minim pentru care
$P(A) = 0$. $P$ se numește polinomul minim al lui $A$. În
cazul matricelor rare, ne așteptam că acest polinom să aibă un grad mai mic
decât cel al polinomului caracteristic. $P$ mereu va divide polinomul
caracteristic, iar din \ref{cayleyhamilton}, gradul acestuia nu va fi mai mare
de $n$. Fie $P = a_{0} + a_{1}x + \ldots + a_{n'}x^{n'}$, unde $n' \leq n$.

\begin{lem}
  \label{polminimdet}
  Dacă $n' = n$, atunci $P$ este chiar polinomul caracteristic al lui $A$, iar
  determinantul este chiar $(-1)^{n}P(0)$.
\end{lem}

\begin{lem}
  Putem folosi polinomul minimal pentru a calcula determinantul matricei $A$.
\end{lem}

\begin{proof}
  Fie $d$ un vector de $n$ elemente
  alese uniform aleator și independent din $F$, iar
  $D =
  \begin{bmatrix}
    d_{1} & 0 & \dots \\
    0 & d_{2} & \dots \\
    \vdots & \ddots & \\
    0 & \dots & d_{n}
  \end{bmatrix}$. Conform lemei Schwartz-Zippel, matricea $AD$ este singulară cu
  probabilitate de cel mult $\frac{n}{|F|}$. În cazul în care nu este, atunci ne
  așteptăm ca polinomul minim să aibă grad egal chiar cu $n$. În acest caz,
  putem calcula rapid $\det(AD)$ folsind \ref{polminimdet} și apoi să aflăm
  $\det(A) = \frac{\det(AD)}{det(D)}$, iar $\det(D) = \prod_{i=1}^{n} d_{i}$.
\end{proof}

Fie $x'$ un vector de $n$ elemente alese uniform aleator și independent din $F$,
iar $x = Ax'$. Fie $S = [x, Ax, A^{2}x, \ldots]$.
Considerăm un alt vector de $n$ elemente, $y$ și următorul șir de valori din $F$: $[yx, yAx, yA^{2}x, \ldots ]$.

Știm că $P(A) = 0$, atunci $\displaystyle\sum\limits_{i=0}^{n'} a_{i} A^{i} = 0$. Obținem că și
$\displaystyle\sum\limits_{i=0}^{n'} y (a_{i} (A^{i} x)) = 0$. Intuitiv, știm că $P(S) = 0$, însă nu
știm să calculam eficient o secvență $b_{0}, b_{1}, \ldots b_{k}$ astfel încât
$\displaystyle\sum\limits_{i=0}^{k} b_{i} S_{i+o} = 0 \ \forall \ o \geq 0$, pentru că elementele din $S$ sunt vectori, însă
atunci când reducem la cazul cu valori, există o șansă destul de mare ca secvență $b$
care anihilează $yS$ să anihileze și $S$, deci
$\displaystyle\sum\limits_{i=0}^{k} b_{i} M^{i}x = 0$.
Înlocuind cu $x = Ax'$, obținem că $M \displaystyle\sum\limits_{i=0}^{k} b_{i} M^{i} x' = 0$, deci
dacă $b_{i} M^{i} x'$ este nenul, atunci am obținem un nucleu al lui $M$.

Pentru a calcula elementele secvenței $S$ se folosesc algoritmi pentru găsirea
\textbf{polinomului minim al unei recurențe liniare}. Algoritmul
\textbf{Berlekamp Massey} dacă $|F|$ este un număr prim (doua corpuri de
aceeași mărime sunt izomorfe), altfel, se factorizează
$|F| = p_{1}^{k_{1}} p_{2}^{k_{2}} \ldots p_{t}^{k_{t}}$ și se rezolvă $t$
probleme cu $|F_{i}'| = p_{i}^{k_{i}}$ cu algoritmul \textbf{Reeds-Sloane}, iar
rezultatele se combina folosind teorema chineză a resturilor. În ce urmează vom
prezenta un alt algoritm echivalent care nu folosește împărțiri, deci nu este
constans la un tip de corp anume \cite{sugiyama}.

\pagebreak

\section{Polinomul minim al unei recurențe liniare}

\begin{defn}
  O secvență infinita $a$ cu elemente dintr-un corp $F$ se numește recurență
  liniară dacă și numai daca exista constantele $c_{1}, c_{2}, \ldots, c_{k}$
  astfel încât
  $a_{t} = a_{t-1}c_{1} + a_{t-2}c_{2} + \ldots + a_{t-k}c_{k} \ \forall \ t > k$.
\end{defn}

\begin{defn}
  Pentru o recurența liniară $a$ și $c_{0}, c_{1}, \ldots c_{k} \in F$, astfel
  încât
  $c_{0} a_{t} = a_{t-1}c_{1} + a_{t-2}c_{2} + \ldots + a_{t-k}c_{k} \ \forall \ t > k$,
  atunci polinomul $c_{0} x^{k} + c_{1} x^{k-1} + \ldots + c_{k} x^{0}$ se
  numește anihilator al lui $a$.
\end{defn}

\begin{defn}
  Un ideal al unui inel $R$ generat de $S \subset R$ este
  $\{f = \displaystyle\sum\limits_{i=1}^{r} x_{i}s_{i} \ | \ r \in \mathbb{N}, \ s_{i} \in S, x_{i} \in R\}$.
\end{defn}

\begin{defn}
  Un ideal $<S>$ dintr-un inel $R$ este de tip finit dacă admite un sistem finit de generatori:
  $\exists \ s_{1}, s_{2}, \ldots s_{r} \in \ $  astfel încât
  $\forall \ s \in \ , s = \displaystyle\sum\limits_{i=1}^{r} x_{i} s_{i}, \ x_{i} \in R$.
\end{defn}

\begin{defn}
  Un ideal se numește principal dacă admite un sistem de generatori format
  dintr-un singur element.
\end{defn}

\begin{defn}
  Inelul $R$ se numește principal dacă orice ideal este principal.
\end{defn}

\begin{thm}
  $F[x]$ este un inel principal.
\end{thm}

\begin{lem}
  Anihilatoarele lui $a$ formează un ideal în $F[x]$.
\end{lem}

\begin{clr}
  Idealul format din anihilatoarele lui $a$ este generat de un polinom monic de
  grad minim. Acesta se numește polinomul minim al lui $a$.
\end{clr}

\begin{thm}
  Funcția generatoare $A(x) = \displaystyle\sum\limits_{n=0}^{\infty} a_{n}x^{n}$ este rațională:
  $A(x) = \frac{P(x)}{Q(x)}$ unde $deg(P) < k$, iar $Q(x) = 1 - \displaystyle\sum\limits_{i=1}^{k} c_{i}x^{i}$.
\end{thm}

\begin{thm}
  Dacă știm că gradul polinomului minim este cel mult $m$, atunci acesta este
  unic determinat de primele $2m$ elemente ale lui $a$; sau altfel spus
  $P(x) \equiv Q(X) A(X) \mod x^{2m}$.
\end{thm}

\noindent \textbf{Identitatea lui Bezout}: dacă $g$ este cel mai mare divizor comun al
polinoamelor $a$ și $b$, atunci există două polinoame $u$ și $v$ astfel încât
$ua + vb = g$.

În \cite{sugiyama} se arată că putem executa algoritmul lui Euclid extins pentru
calcularea valorilor $u$ și $v$, alegând $a = A(x) \mod x^{2m}$, iar
$b = x^{2m}$ până când $deg(g) < m$.

\pagebreak

\section{Filtrare prin rădăcini ale unității}

\label{rootsofunityfilter}

\textbf{Filtrarea prin rădăcini ale unității} \cite{rootsofunityfilter} este o tehnică de izolare și
însumare a coeficienților unui polinom. Aceasta se pretează, în particular,
atunci când indexurile sunt periodice modulo $n$.

\begin{thm}
  Dacă $p$ este un polinom și $\zeta$ este o rădăcină a unității de ordin
  $n$ ($\zeta^{n} = 1$), atunci $\frac{1}{n} \displaystyle\sum\limits_{i=0}^{n-1} p(\zeta^{i})$
  este suma indexurilor dizibile prin $n$.
\end{thm}

\begin{proof}
  Fie $p = a_{0}x^{0} + a_{1}x^{1} + \ldots + a_{k}x^{k}$. Consideram contribuția
  unui index $p$, nedivizibil prin $n$ la suma
  $\displaystyle\sum\limits_{i=0}^{n-1} p(\zeta^{i}) = a_{p}(\zeta^{0} + \zeta^{p} + \zeta^{2p} + \ldots + \zeta^{(n-1)p})$.
  Termenul din partea dreaptă este de fapt $\displaystyle\sum\limits_{i=0}^{n-1} \zeta^{i*p} = \frac{\zeta^{p*n} - 1}{\zeta^{p} - 1} = 0$,
  deci se anulează. Celelalte indexuri vor avea coeficient $1$ pentru fiecare
  $\zeta^{i}$, deci se vor aduna în total de $n$ ori, motiv pentru care toată suma
  se împarte la $n$.
\end{proof}


\pagebreak

\section{Rădăcini primitive}
\label{primitiveroot}

\cite{primitiveroot} În ce urmează presupunem că lucrăm într-un corp finit $\mathbb{F}_{p}$, unde $p$ este un număr prim.

\begin{defn}
  Un număr $g$ se numește rădăcină primitiva modulo $n$ dacă și numai dacă pentru orice $a$ coprim cu $p$ există un alt număr
  $l$ astfel încât $g^{l} = a \mod p$.
\end{defn}

\begin{thm}
  \label{carmichael}
  (Funcția Carmichael) Fie $g$ o rădăcină primitivă, atunci cel mai mic număr $l$ astfel încât $g^{l} = 1$ este $\phi(p)$.
\end{thm}

\begin{thm}
  \label{carmichaelr}
  Reciproca teoremei \ref{carmichael} este adevărată și stă la baza algoritmului pentru aflarea generatorului.
\end{thm}

Un algoritm evident pentru a caută rădăcini primitive încearcă numerele din $\mathbb{F}_{p}$ pe rând și apoi le calculează
puterile pentru a verifica dacă sunt diferite. Deși în general rădăcină primitivă este un număr destul de mic, acest algoritm
nu este cel mai eficient posibil.

Din \ref{carmichaelr} știm că, pentru un $a$ fixat, dacă găsim un $l < \phi(p)$ pentru care $a^{l} = 1$, atunci acesta nu este un
generator. Din \textbf{teorema lui Euler} știm că orice $a$ coprim cu $p$ satisface $a^{\phi(p)} = 1 \mod p$. Aceste observații pot
îmbunătăți algoritmul menționat anterior, însă, pentru că noi îl folosim pentru $p$ număr prim, atunci $\phi(p) = p - 1$, deci nu
ajută foarte mult. \textbf{Teorema lui Lagrange} mai spune că este suficient să verificam că $a^{d} \neq 1 \mod p$ pentru toți $d \ |\ \phi(p)$.
Aceasta este deja o îmbunatățire foarte mare, deoarece numărul de divizori ai lui $\phi(p)$ este de ordinul $O(p^{\frac{1.066}{\ln \ln p}})$.

Dacă $\phi(p) = a_{1}^{b_{1}} a_{2}^{b_{2}} \ldots a_{t}^{b_{t}}$, atunci este de ajuns să se verifice acei $d = \frac{\phi(p)}{a_{i}}$.

\begin{proof}
  Presupunem că există un divizor propriu $d$ al lui $\phi(p)$ astfel încât $g^{d} = 1 \mod p$. Atunci există un factor prim $a_{i}$
  din factorizarea lui $\phi(p)$ astfel încât $d \ | \ \frac{\phi(p)}{a_{i}}$ și evident $g^{\frac{\phi(p)}{a_{i}}} = 1 \mod p$.
\end{proof}

Complexitatea de timp a acestei soluții este acum $O(g \log^{2} p)$, iar dacă presupunem ipoteza Riemann ca fiind adevărată, $g = O(\log^{6} p)$.

  \chapter{Probleme}

\section{Cuplaj cu costuri pe noduri}

\noindent \textbf{Enunt.} Se da un graf bipartit $G(U + V, E)$ si o functie de cost $w \colon U \to \mathbb{R}$. Sa se determine cuplajul
maximal care maximizeaza suma costurilor nodurilor alese in cuplaj.

\noindent \textbf{Solutie.} Este corect sa prioritizam nodurile in ordinea descrescatoare a costurilor asociate.

\pagebreak

\section{Cuplaj cu costuri pe muchii}

\noindent \textbf{Enunt.} Se da un graf bipartit $G(U + V, E)$ si o functie de cost $w \colon E \to \mathbb{R}$. Sa se determine cuplajul
\textbf{perfect} care maximizeaza suma costurilor muchiilor alese in cuplaj.

\noindent \textbf{Solutie.} Consideram urmatoarea formulare a problemei ca una de programare liniara (\cite{assignmentlp}):

\begin{equation*}
\begin{array}{ll@{}ll}
  \text{max}  & \displaystyle\sum\limits_{(x, y) \in E} w((x, y))m_{x, y} &\\
  \text{s.t.} & \displaystyle\sum\limits_{y} m_{x, y} = 1 \ \forall \ x &\\
              & \displaystyle\sum\limits_{x} m_{x, y} = 1 \ \forall \ y &\\
              & m_{x,y} \in \{0, 1\} \ \forall x, y
\end{array}
\end{equation*}

Aceasta este o problema discreta pentru ca am conditionat ca $m_{x, y}$ sa fie $0$ sau $1$, dar se poate arata ca in cazul problemei cuplajului
de cost maxim putem relaxa aceasta conditie la varianta continua si anume $0 \leq m_{x, y} \leq 1$. \textbf{Teorema Birkhoff-von Neumann} spune
ca fiecare cuplaj fractionar poate fi descompus ca o combinatie convexa de cuplaje. Sa consideram formularea duala a acestei probleme:

\begin{equation*}
\begin{array}{ll@{}ll}
  \text{min}  & \displaystyle\sum\limits_{x} p_{x} + \displaystyle\sum\limits_{y} q_{y} &\\
  \text{s.t.} & p_{x} + q_{y} \leq w((x, y)) \ \forall \ (x, y) \in E &\\
              & \displaystyle\sum\limits_{y} q_{y} m_{x, y} = q_{y} \ \forall \ x &\\
              & \displaystyle\sum\limits_{x} p_{x} m_{x, y} = p_{x} \ \forall \ y &\\
\end{array}
\end{equation*}

Duala unei probleme de maximizare este una de minimizare, iar optimul ei consistuie o limita superioara pentru problema primala
\textbf{(dualitate usoara)}. Atunci cand este chiar egala, se numeste \textbf{dualitate puternica}.

\begin{thm}
  Problema cuplajului perfect de cost maxim admite dualitate puternica.
\end{thm}

\begin{thm}
  Duala problemei cuplajului maximal este acoperirea muchiilor cu numar minim de noduri.
\end{thm}

\begin{proof}
  Daca am formula problema cuplajului maximal ca o problema de programare liniara atunci golul ar fi sa maximizam numarul de muchii alese,
  cu restrictia ca fiecare nod trebuie sa aiba cel mult o muchia aleasa incidenta. Folosind iar \textbf{teorema Birkhoff-von Neumann} putem
  arata ca putem relaxa problema dintr-una discreta intr-una continua. Duala acestei probleme este sa se minimizeze numarul de noduri alese
  cu restrictia ca fiecare muchie trebuie sa aiba macar un nod ales. Aceasta este chiar problema acoperirii muchiilor cu numar minim de noduri.
\end{proof}

\textbf{Algoritmul ungar} este un algoritm care rezolva problema cuplajului de cost maxim, folosindu-se de dualitatea prezentata. Acesta
construieste simultan un cuplaj perfect de cost minim si solutia duala a caror valoare este egala cu costul cuplajului. Se pastreaza
urmatorii invarianti:

\begin{itemize}
  \item Fiecare muchie $(x, y)$ satisface $p_{x} + q_{y} \geq w((x, y))$. Daca $p_{x} + q_{y} = w((x, y))$, atunci $(x, y)$ este o muchie stransa.
  \item Muchiile alese in cuplaj sunt si stranse.
  \item Dupa fiecare faza a algoritmului, marimea cuplajului creste cu $1$ pana ajunge la cel perfect.
\end{itemize}

\noindent Algoritmul urmareste acesti pasi (\cite{hungarian}):

\begin{enumerate}
  \item Initializam $p_{x} = 0 \ \forall \ x$ si $q_{y} = \max w((x, y)) \ \forall \ (x, y) \in E \ \forall \ y$.
  \item Se cauta cuplajul maximal folosind doar muchii stranse.
  \item Daca cuplajul este perfect algoritmul se termina. Altfel, folosind cuplajul calculat, se calculeaza dualul $V$,
    adica acoperirea muchiilor cu numar minim de noduri.
  \item Se calculeaza $\bigtriangleup = \text{min}_{(x, y) \in E, x \notin V, y \notin V} p_{x} + q_{y} - w((x, y))$.
  \item Potentialii se recalculeaza dupa urmatorea regula:
    \begin{itemize}
      \item $p_{x} = p_{x} + \bigtriangleup \ \forall \ x \in V$
      \item $q_{y} = q_{y} - \bigtriangleup \ \forall \ y \notin V$
    \end{itemize}
  \item Repeta pasul 2.
\end{enumerate}

Se observa ca algoritmul ungar foloseste algoritmul de cuplaj maximal ca un oracol. In lucrarea initiala se foloseste un algoritm bazat pe
ideea de la algoritmii de flux maxim, incrementand dimensiunea cuplajului mereu cu 1, gasind o cale alternanta $P$ si inlocuind cuplajul $M$
cu $M \oplus P$. Insa aici putem folosi metodele algebrice studiate.

Pentru a calcula cuplajul maximal este de ajuns sa gasim rangul matricei Edmonds asociata grafului, adica marimea bazei formata din spatiul
liniar al vectorilor de pe linii. Acestea vor fi chiar nodurile alese in cuplaj din $U$.

Ca sa obtinem acoperirea minima a muchiilor cu numar minim de noduri, acestea sunt chiar nodurile aflate la distanta impara
de vreun nod necuplat din $U$.

\pagebreak

\section{Izomorfism de subarbori}
\begin{tabular}{l@{\extracolsep{1cm}}l}
  Concurs: & Lot Râmnicu Vâlcea 2015 - Baraj 2 Seniori\\
  Limita de timp: & 0.3\ s\\
  Limita de memorie: & 64\ MB\\
\end{tabular}

\hspace{1cm}

\noindent \textbf{Enunt.} Se dau doi arbori cu radacina in nodul $0$ si cel mult $500$ de noduri, $A$ si $B$.
O operatia consta in stergerea unei frunze. Care este numarul minim de operatii aplicate pe $A$ astfel incat
sa devina izomorf cu $B$?

\noindent \textbf{Solutie.} Cei doi arbori sunt izomorfi daca exista o re-etichetare a nodurilor in asa fel incat sa fie identici.
Consideram mai intai o problema mai simpla: cum vedem daca doi arbori sunt izomorfi? Vom atribui fiecarui arbore un polinom in mai
multe variabile si vom reduce problema la una de echivalenta de polinoame.

\begin{figure}[H]
  \centering
  \begin{tikzpicture}[level distance=1.5cm,
    level 1/.style={sibling distance=3cm},
    level 2/.style={sibling distance=1.5cm}]
    \node {$(z + (x + 1)^{2})(z + y + 1)$}
      child {node {$(x + 1)^{2}$}
        child {node {$1$}}
        child {node {$1$}}
      }
      child {node {$y + 1$}
      child {node {$1$}}
      };
  \end{tikzpicture}
  \caption{Constructia polinomului}
\end{figure}

\begin{algorithm}[H]
  \DontPrintSemicolon
  \SetKwFunction{FMain}{DFS-POLY}
  \SetKwProg{Fn}{Function}{:}{}
  \Fn{\FMain{$T$, $u$, $p$}}{
    poly := 1\;
    \For{$v \in T_{u}$}{
      \If{$v \neq p$}{
        poly := poly * ($x_{u}$ + DFS-POLY(T, v, u))\;
      }
    }
    \KwRet poly\;
  }
  \;
\end{algorithm}

Pentru doi arbori, daca evaluam polinoamele cu valori uniform aleatoare alese din $\mathbb{F}_{p}$, atunci probabilitatea de coliziune este de
cel mult $\frac{N}{p}$, unde $N$ este numarul de noduri, conform lemei \textbf{Schwartz-Zippel}. Doi arbori ne-izomorfi vor avea polinoamele diferite
pentru ca inelul polinoamelor $\mathbb{F}_{p}[x_{1}, x_{2}, \ldots, x_{N}]$ determina o factorizare unica a polinomului, descompus intr-un produs de
polinoamele ireductibile si constante.

Avand acest algoritm, putem construi o solutie cu backtracking in care pornind din frunze, eliminam o multime de noduri, mai exact atatea
noduri cat are $A$ in plus fata de $B$ si apoi verificam izomorfismul cu algoritmul dat.

In continuare, observam din solutia precedenta ca raspunsul este fix $|A| - |B|$, daca exista o secventa de operatii. Vom nota cu $a_{v_{1}}$
subarborele nodului $v_{1}$ din $A$ si cu $b_{v_{2}}$ subarborele nodului $v_{2}$ din $B$. Vom construi urmatoarea recurenta folosind
programare dinamica: $d_{v_{1}, v_{2}}$ este adevarat daca putem executa o secventa de operatii astfel incat sa transformam subarborele
nodului $v_{1}$ intr-unul izomorfm subarborelui nodului $v_{2}$ (din celalalt arbore). Este clar ca pentru perechi unde $|a_{v_{1}}| < |b_{v_{2}}|$ valoarea va fi fals. Altfel, consideram fii directi ai acestor doua noduri: $c_{1, 1}, c_{1, 2}, \ldots, c_{1, n_{1}}$ si $c_{2, 1}, c_{2, 2}, \ldots, c_{2, n_{2}}$.
Daca $n_{1} < n_{2}$, atunci nu se poate sa eliminam noduri, micsorand astfel $n_{1}$, incat sa obtinem $n_{2}$. In ce urmeaza ne intereseaza sa gasim o
submultime a fiilor primului nod $c_{1, i_{1}}, c_{1, i_{2}}, \ldots, c_{1, i_{n_{2}}}$ astfel incat $d_{c_{1, i_{j}}, c_{2, j}}$ sa fie adevarate
pentru toti $1 \leq j \leq n_{2}$.

O prima solutie ar fi sa consideram toate cele $\binom{n_{1}}{n_{2}}$ solutii posibile. Aceasta solutie este deja mai eficienta decat prima propusa.

\begin{thm}
  \label{hall}
  \textbf{Hall.} Fie $G = (U \cup V, E)$ un graf bipartit. Exista un cuplaj care acopera $U$ daca si numai daca pentru orice submultime $W$
  a lui $U$, daca consideram multimea $N_{G}(W)$ ca fiind reuniunea vecinilor nodurilor din $W$, atunci $|W| \leq N_{G}(W)$.
\end{thm}

Cum putem folosi \textbf{teorema lui Hall} in cazul nostru este sa construim matricea $d'$ asociata vecinilor lui $v_{1}$, respectiv $v_{2}$
unde $d'_{i, j} = d_{c_{1, i}, c_{2, j}}$. Daca pentru fiecare multime de coloane consideram multimea liniilor care au macar o valoare de adevarat
pe vreuna din coloane, atunci vrem ca numarul de linii sa nu fie mai mic decat numarul de coloane. Aceasta solutie din pacate ramane exponentiala
si se poate construi in timp $O(2^{n_{2}} n_{1})$ daca se proceseaza submultimile de coloane in ordinea codurilor Gray, in timp ce solutia evidenta ar lua
$O(2^{n_{2}} n_{1}n_{2})$. O proprietate inedita a codurilor Gray este ca mastile consecutive de biti vor fi diferite pe fix un bit. Astfel vom determina acel bit,
asociat unei coloane si apoi vom folosi un vector de frecventa pentru a determina ce linii mai raman active pentru noua submultime de coloane, in timp
cel mult $O(n_{1})$.

Problema pe matricea $d'$ este chiar de cuplaj in graf bipartit, iar $d'$ este chiar textbf{matricea Edmonds} (\ref{edmonds}). Ramane sa ii calculam
rangul si sa verificam ca este egal cu $n_{2}$. Complexitatea acestei solutii este:

\begin{equation}
  \displaystyle\sum\limits_{v_{1} \in A} \displaystyle\sum\limits_{v_{2} \in B} n_{1}n_{2}^{2} = \displaystyle\sum\limits_{v_{1} \in A} n_{1} \displaystyle\sum\limits_{v_{2} \in B} n_{2}^{2} \leq |A||B|^{2}
\end{equation}

\noindent pentru ca $a^{2} + b^{2} \leq (a + b)^{2}$.

\pagebreak

\section{Acoperire cu cai a unui graf orientat aciclic}

\noindent \textbf{Enunt.} Se da un graf orientat aciclic $G(V, E)$. Sa se gaseasca o multime de cardinal minim de cai, astfel incat
fiecare nod sa faca parte din exact o cale.

\noindent \textbf{Solutie.} Un graf orientat aciclic este un graf orientat fara cicli orientati. Acesta admite o sortate topologica,
adica o ordonare a nodurilor $v_{1}, v_{2}, \ldots$ in asa fel incat daca exista arcul $(v_{i}, v_{j})$, atunci $v_{i}$ apare inaintea
lui $v_{j}$ in ordonare. Algoritmul pentru a calcula sortarea topologica este urmatorul:

\begin{algorithm}[H]
  \DontPrintSemicolon
  \SetKwFunction{FMain}{DFS-VISIT}
  \SetKwProg{Fn}{Function}{:}{}
  \Fn{\FMain{$G$, $color$, $u$, $T$}}{
    \If{$color_{u}$ = BLACK}{
      \KwRet T\;
    }
    $color_{u}$ := GRAY\;
    \For{$v \in G_{u}$}{
      T := DFS-VISIT($G, color, v, T$)\;
    }
    $color_{u}$ := BLACK\;
    \KwRet [u] + T\;
  }
  \;
\end{algorithm}

\begin{algorithm}[H]
  \DontPrintSemicolon
  \SetKwFunction{FMain}{DFS}
  \SetKwProg{Fn}{Function}{:}{}
  \Fn{\FMain{$G$}}{
    \For{$u \in G$}{
      $color_{u}$ := WHITE\;
    }
    T := []\;
    \For{$u \in G$}{
      T := DFS-VISIT(G, color, u, T)\;
    }
    \KwRet T\;
  }
  \;
\end{algorithm}

\pagebreak

Algoritmul practic construieste inversa postordinii si aceasta este chiar o sortare topologica posibila.

Avand calculata sortarea topologica, o prima solutie cu backtracking va considera nodurile in ordinea inversa.
Ne putem imagina caile ca o colorare, iar acum la fiecare pas al algoritmului trebuie sa stabilim culoarea pentru
nodul curent. Pentru ca le procesam in ordinea inversa sortarii topologice, stim ca am procesat deja toti vecinii lui
iar acestia sunt colorati in culorile $\{C_{1}, C_{2}, \ldots, C_{k}\}$. Pentru a asigura invariantul de cale, odata ce
un nod preia culoarea unui vecin, vecinul isi pierde culoarea si nu mai poate fi ales pe masura ce coboram in adancime.
Atunci, pentru nodul curent trebuie sa alegem una din aceste culori sau sa cream o culoare noua. Dupa ce toate nodurile
au fost colorate putem sa ne uitam la numarul de culori diferite folosite si aceasta va fi un candidat pentru raspunsul
problemei.

\begin{lem}
  Complexitatea acestei solutii este de $O(|E|2^{|V|})$.
\end{lem}

\begin{proof}
  In cel mai rau caz, consideram nodurile in ordine si la fiecare pas avem o decizie, ne atasam la o cale existenta sau cream o cale
  noua.
\end{proof}

\noindent Solutia polinomiala a problemei nu este evidenta, dar este sugerata de conexiunea cuplajului in graf bipartit cu teorema lui
Dilworth. Vom numi o \textbf{anticale} o secventa de noduri $v_{1}, v_{2}, \ldots, v_{k}$ pentru care
$\forall \ v_{i}, v_{j} \ (v_{i}, v_{j}), (v_{j}, v_{i}) \notin E$.

\begin{thm}
  \label{Dilworth}
  \textbf{Dilworth.} Numarul minim de cai necesare ca sa acoperi graful este egal cu marimea celei mai lungi anticai.
\end{thm}

\begin{thm}
  \label{Konig}
  \textbf{Kőnig (1931).} In orice graf bipartit, numarul de muchii dintr-un cuplaj maximal este egal cu numarul de noduri din acoperirea minima.
\end{thm}

Avand aceasta observatie, este mai clara reducerea la cuplaj in graf bipartit. Construim graful bipartit in care dublam fiecare nod din graful
initial astfel incat sa apara in ambele multimi. Apoi, muchiile din graful initial le vom considera si in acesta, doar ca vor conecta doar
noduri din multimi diferite. Acum, daca am gasi o multime maxima de noduri independente, aceasta va fi o \textbf{anticale}.

\pagebreak

\section{Bilingual}

\begin{tabular}{l@{\extracolsep{1cm}}l}
  Concurs: & Google Code Jam 2015 Round 2\\
  Limita de timp: & 10\ s\\
  Limita de memorie: & 256\ MB\\
\end{tabular}

\hspace{1cm}

\noindent \textbf{Enunt.} Se dau $N + 2 \ (2 \leq N \leq 200)$ liste de cuvinte. Se stie ca prima lista contine doar cuvinte in limba
engleza, a doua contine doar cuvinte in limba franceza. Restul de $N$ nu se cunosc si trebuie catalogate intr-una
din cele doua limbi in asa fel incat sa se minimizeze numarul de cuvinte care fac parte in ambele limbi.

\hspace{1cm}

\noindent \textbf{Solutie.} O prima solutie cu backtracking trebuie sa asocieze fiecarei liste o limba si apoi sa
verifice numarul de cuvinte care se afla in ambele limbi. Sunt $2^{N}$ posibilitati de a alege, iar verificarea poate
dura si pana la suma numarului de cuvinte, daca acestea se normalizeaza in prealabil. Sunt mai multe moduri in care
putem imbunatati constanta acestei solutii. In primul rand, putem ignora cuvintele care nu apar in mai mult de o
propozitie. De asemenea, in cazul acestui backtracking, o alta optimizare semnificativa ar fi iterative deepening pe raspuns,
adica sa nu urmarim ramuri care cresc numarul de cuvinte incompatibile din raspuns imediat ce le observam (depth-first),
dar sa o abordam gradual si sa preferam o crestere in latime, nu in adancime. Din pacate aceasta solutie nu este destul
de rapida, nici macar pentru anul $2015$, cand solutiile la Code Jam se rulau local si se cerea doar rezultatul unei
rulari de program.

Problema se aseamana unei probleme clasice de taietura minima, numita \textbf{Image segmentation}. Taietura minima este duala
problemei de flux maxim. In aceasta problema sunt $N$ pixeli. Fiecarui pixel $i$ se poate asocia o culoare de prim plan de cost
$f_{i}$ sau o culoare de fundal $b_{i}$. Daca doi pixeli $i$ si $j$ sunt adiacenti si au asocieri diferite, atunci trebuie scazuta
o valoare $p_{i,j}$. Problema cere sa se efectueze asocierile in asa fel incat sa maximizeze costul. Fie $P$ multimea pixelilor carora
le-a fost asociat prim-planul si $Q$ multimea pixelilor carora le-a fost asociat fundalul, atunci dorim sa maximizam
$\displaystyle\sum\limits_{i \in P} f_{i} + \displaystyle\sum\limits_{i \in Q} b_{i} - \displaystyle\sum\limits_{i \in P, j \in Q \lor i \in Q, j \in P} p_{i,j}$. Aceasta se poate reformula ca o problema
de minimizare a valorii $\displaystyle\sum\limits_{i \in P, j \in Q \lor i \in Q, j \in P} p_{i,j}$. Reteaua de flux maxim dubleaza nodurile pentru fiecare pixel
in asa fel incat un nodul $2u$ inseamna $u$ a fost asociat pe prim-plan, iar $2u+1$ inseamna $u$ a fost asociat pe fundal. Sursa este
conectata catre toate nodurile de prim-plan $2u$ cu capacitate $f_{u}$, iar destinatia este conectata catre toate nodurile de fundal $2u+1$
cu capacitate $b_{u}$. Muchii de capacitate $p_{i,j}$ se adauga intre nodurile adiacente de asocieri diferite. A taietura de la sursa la
destinatie este apoi o asociere a pixelilor in multimile $P$, respectiv $Q$.

In cazul nostru, culoarea de prim plan inseamna un cuvant in limba engleza, iar culoarea de fundal inseamna un cuvant in limba franceza.
Observam ca nu ne intereseaza valorile $f_{i}, b_{i}$, trebuie doar sa fie suficent de mari, iar valoarea de ``penalty'' se aplica atunci cand
un cuvant este asociat in ambele limbi si are cost $1$. Trebuie totusi sa avem grija sa nu atribuim mai multe limbi aceleiasi liste, asadar mai
asociem un cost $\inf$ intre cuvinte de limba diferita din aceeasi propozitie. Trebuie tratate cu atentie cuvintele din primele doua propozitii.

In schimb, nu am prezentat degeaba aceasta problema. Ea poate fi rezolvata mai departe si ca o problema de cuplaj maxim in graf bipartit.
Putem imparti cuvintele in trei multimi, cuvintele care sunt doar in engleza, cuvintele care sunt doar in franceza si cuvintele
care sunt in ambele limbi. Pentru a minimiza numarul de cuvinte din ambele limbi ne dorim sa maximizam numarul de cuvinte care sunt intr-o
singura limba. Ca inainte, vom construi doua noduri pentru fiecare cuvant $u$, nodul $2u$ va fi ales daca $u$ nu este un cuvant in engleza,
iar nodul $2u + 1$ va fi ales daca $u$ nu este un cuvant in franceza. Pentru fiecare doua cuvinte $u_{1}, u_{2}$ din aceeasi lista adaugam o
muchie intre $2u_{1}$ si $2u_{2} + 1$. Ramane de gasit cea mai mare multime de noduri astfel incat nu exista o muchie intre nicio pereche, adica
\textbf{maximal independent set}. Pe caz general aceasta nu se poate rezolva polinomial, insa graful creat este evident unul bipartit. In grafurile
bipartite \textbf{maximal independent set} este complementul acoperirii minime cu nodurilor (\textbf{minimal vertex cover}).

Asadar, folosind \ref{Konig}, putem folosi algoritmul de cuplaj maximal \ref{Algoritm paralelizabil pentru cuplaj} pentru a calcula raspunsul problemei.

\pagebreak

\section{Paritatea numarului de cuplaje}
\begin{tabular}{l@{\extracolsep{1cm}}l}
  Concurs: & Olimpiada nationala pentru studenti 2015, Runda finala\\
  Limita de timp: & 1\ s\\
  Limita de memorie: & 20\ MB\\
\end{tabular}

\hspace{1cm}

\noindent \textbf{Enunt.} Se da un graf bipartit $G = (U + V, E)$ unde $|U|, |V| \leq 100$. Sa se determine paritatea numarul de cuplaje perfecte.

\hspace{1cm}

\noindent \textbf{Solutie.} Este clar ca daca $|U| \neq |V|$ atunci numarul este $0$, deci par. Altfel, putem pentru inceput sa testam fiecare permutare, in complexitate $O(|U|!)$.

Putem imbunatati solutia precedenta daca o abordam cu programare dinamica. Construim tabela $D_{i, S}$ care este paritatea numarului de cuplaje a
primelor $i$ elemente din $U$ cu multimea de noduri din $S \subseteq V$. Recurenta se poate construi astfel:

\begin{equation}
  D_{i, S} = \displaystyle\sum\limits_{(u_{i}, v_{j}) \in E} D_{i-1, S - \{v_{j}\}} \mod 2
\end{equation}

\noindent Aceasta solutie are complexitate de timp $O(2^{|U|} (|U| + |E|))$

\begin{thm}
  Numarul de cuplaje perfecte este egal cu permanentul matricei Edmonds asociat.
\end{thm}

\begin{proof}
  Permanentul unei matrice patratica $A$ de marime $N \times N$ este egal cu $\displaystyle\sum\limits_{P \in S_{N}} \prod_{i=1}^{N} A_{i, P_{i}}$.
  Valoarea $\prod_{i=1}^{N} A_{i, P_{i}}$ este $1$ daca si numai daca $P$ este un cuplaj perfect. Insumam aceasta valoare
  pentru toate permutarile. Deci am numarat cuplajele perfecte.
\end{proof}

Se observa, din formula, ca permanentul este determinantul ``fara semn''. Totusi, atunci cand lucram in $\mathbb{F}_{2}$,
semnul nu conteaza, pentru ca $-1 \mod 2 = 1$. Asadar, pentru a afla paritatea numarului de cuplaje, este de ajuns sa
determinam paritatea determinatului. Pentru aceasta, putem folosi metoda Eliminarii Gauss in $\mathbb{F}_{2}$, in complexitate
$O(\frac{N^{3}}{w})$, unde $w$ este marimea unui cuvant. Pana acum nu se cunoste vreun algoritm polinomial pentru a calcula
permanentul pe caz general.

\pagebreak

\section{Divisible Matching}

\begin{tabular}{l@{\extracolsep{1cm}}l}
  Concurs: & CS Academy Round \#67\\
  Limita de timp: & 4\ s\\
  Limita de memorie: & 256\ MB\\
\end{tabular}

\hspace{1cm}

\noindent \textbf{Enunt.} Se da un graf bipartit $G(U \cup V, E)$ cu $1 \leq |U| = |V| \leq 100$,
un numar intreg $2 \leq K \leq 100$ si o functie pentru a determina valoarea unei muchii
$f : E \to \{0, 1, \ldots, K-1\}$. Sa se determine daca exista un cuplaj perfect $M$,
astfel incat $K \ | \ \displaystyle\sum\limits_{e \in M} f(e)$.

\hspace{1cm}

\noindent \textbf{Solutie.} Fara a pierde din generalitate, presupunem ca $|U| = |V| = N$.
O solutie evidenta in complexitate $O(N!)$ construieste si verifica toate permutarile din $S_{N}$.
Chiar daca le procesam in ordine aleatoare, numarul asteptat de pasi este tot de ordinul $N!$. \\
Urmatoarea solutie pe care o putem aborda este sa optimizam solutia de backtracking mentionata anterior
cu metoda programarii dinamice. Construim tabela $D_{i,\text{rem}, S}$ ca fiind o valoare booleana daca poti
cupla nodurile $\{u_{1}, u_{2}, \ldots, u_{i}\}$ din $U$ cu nodurile $S \subseteq V$ astfel incat restul sumei
valorilor muchiilor alese de pana acum, modulo $K$ este rem. Recurenta cu care se poate construi este:

\begin{equation}
  D_{i, \text{rem}, S} = \bigvee_{j \in S \wedge e_{i, j} = (u_{i}, v_{j})} D_{i-1, \text{rem} - f(e_{i, j}), S - \{\j\}}
\end{equation}

\noindent Din pacate solutie ramane exponentiala, complexitatea acesteia fiind $O(KN^{2}2^{N})$.

\pagebreak

In ce urmeaza ne putem gandi sa rezolvam o subproblema mai simpla, mai exact cea in care $K=2$.
Ne intereseaza un cuplaj cu numar par de muchii de $1$. Daca privim muchiile cu $1$ ca fiind
rosii, iar celelalte ca fiind albastre, atunci obtinem problema \textbf{cuplajului rosu-albastru} (\ref{redbluematching}),
in care ne intereseaza doar ca numarul de muchii rosii sa fie par. In ce urmeaza ne vom concentra pe varianta
in care folosim lema Schwartz-Zippel. Avand \textbf{matricea Edmonds} $A$ construita special pentru acest caz,
$\det(A)$ este un polinom monic in variabila $y$. Atunci cand voiam sa aflam daca exista un cuplaj cu numar
fix de muchii rosii, interpolam $\det(A)$ si verificam daca $y^{k}$ are un coeficient nenul. In cazul de fata
ne intereseaza daca exista macar un termen $y^{k}$ cu coeficient nenul, iar $k$ par. O prima varianta este sa
interpolam polinomul pentru fiecare numar par, insa putem face mai eficient. O metoda sa aflam suma coeficientilor
pari ai unui polinom $p$ este sa evaluam $\frac{p(1) + p(-1)}{2}$. Desigur, in cazul nostru, acest calcul se
efectueaza tot intr-un corp finit si putem sa obtinem iar $0$ pentru ca s-au anulat coeficientii adunatii.
Din fericire, putem aplica iar lema Schwartz-Zippel sa obtinem ca acest fapt este, in continuare, indeajuns
de improbabil.

Pentru a continua sa rezolvam problema pentru orice $K$, amintim ca insumarea coeficientilor divizibil cu o anumita valoare
se poate face folosind \textbf{filtrarea prin radacini ale unitatii} (\ref{rootsofunityfilter}). Problema este insa
sa gasim un corp potrivit care are radacini ale unitatii de ordinul $K$. Daca nu gasim in timp util, mai putem sa extindem
corpul in asa fel incat sa introducem $\zeta \neq 1$ pentru care $\zeta^{K} = 1$. Insa, daca abordam asa, operatiile
aritmetice vor dura $O(K^{2})$ (sau macar $O(K \log K)$ daca folosim algoritmi de aritmetica "state of the art"),
in loc de $O(1)$, unde tinteam. Corpul pe care il cautam va fi $\mathbb{F}_{p}$, unde $p$ este un numar prim pentru care
$p \mod k = 1$. Pe modelul RAM operatiile aritmetice in $\mathbb{F}_{p}$ dureaza $O(1)$. Presupunem ca generatorul acestui
corp este $g$, pe care il putem gasi cu \textbf{algoritmul pentru radacini primitive} \ref{primitiveroot}. Stim ca
$\phi(p) = p - 1$, deci $g^{p - 1} = 1 \mod p$, iar $k\ |\ p - 1$, deci $\zeta = g^{\frac{p-1}{k}}$.

Complexitatea acestei solutii este acum $O(K W(N))$, unde $W(N)$ este timpul necesar calculului de determinant si este
de ajuns sa se incadreze in limitele date.

\section{Xor Matching}

\begin{tabular}{l@{\extracolsep{1cm}}l}
  Concurs: & Codechef September Challenge 2018 Division 1\\
  Limita de timp: & 2\ s\\
  Limita de memorie: & 2\ GB\\
\end{tabular}

\hspace{1cm}

\noindent \textbf{Enunt.} Se da o matrice $A$ de marime $N \times N \ (1 \leq N \leq 60)$ cu valori intregi in celule
din multimea $\{0, 1, \ldots, 1023\}$. Se considera toate permutarile de coloana ale matricei $A$ si se calculeaza
$A_{1, 1} \oplus A_{2, 2} \oplus \ldots \oplus A_{N, N}$, unde $\oplus$ este operatie de xor pe biti.
Sa se determine ce valori se pot obtine.

\hspace{1cm}

\noindent \textbf{Solutie.} O prima solutie este sa consideram toate cele $N!$ permutari ale coloanelor si sa se calculeze
valorile.

O imbunatatire a solutiei precedente este sa aplicam programare dinamica. Mai exact, vom contrui recurenta $D_{i, \text{sum}, S}$
ca fiind o valoare booleana care indica daca putem permuta pentru primele $i$ linii coloanele din multimea $S$ astfel incat sa
suma xor pana acum sa fie egala cu $\text{sum}$. Recurenta este urmatoarea:

\begin{equation}
  D_{i, \text{sum}, S} = \bigvee_{j \in S} D_{i, \text{sum} \oplus A_{i, j}, S - \{j\}}
\end{equation}

Solutia ramane apoi in acele valori adevarate $\text{sum}$ din $D_{N, \text{sum}, \{1, 2, 3, \ldots, N\}}$. Complexitatea acestei solutii
este $O(N^{2}2^{N}\max(A))$, care din pacate nu este destul de rapida pentru restrictiile date.

La prima vedere nu este evidenta relatia pe care problema o are cu cuplajul in grafuri bipartite. Insa daca privim liniile matricei ca
o secventa de noduri $u_{1}, u_{2}, \ldots, u_{N}$, iar coloanele $v_{1}, v_{2}, \ldots, v_{N}$, atunci $A$ este matricea de adiacenta a
unui graf bipartit in care exista o muchie intre oricare doua noduri $u_{i}, v_{j}, 1 \leq i, j \leq N$ de valoare $A_{i, j}$.

Vom incerca sa rezolvam o problema mai simpla acum, mai exact cazul cand $\max(A) = 1$. In acest caz, putem obtine suma xor $0$ daca exista
un cuplaj cu numar par de muchii de valoare $1$, iar suma $1$ daca exista un cuplaj cu numar impar de muchii de valoare $1$. Aceasta subproblema
aduce aminte la problema cuplajului rosu-albastru (\ref{redbluematching}). Ca pana acum, vom considera varianta cu lema Schwartz-Zippel pentru
simplitatea operatiilor aritmetice. Avand \textbf{matricea Edmonds} $A$ construita special pentru acest caz,
$\det(A)$ este un polinom monic in variabila $y$. Atunci cand voiam sa aflam daca exista un cuplaj cu numar
fix de muchii rosii, interpolam $\det(A)$ si verificam daca $y^{k}$ are un coeficient nenul. In cazul de fata
ne intereseaza daca exista macar un termen $y^{k}$ cu coeficient nenul, iar $k$ e par pentru cazul cu suma
xor $0$, iar impar pentru cazul cu suma xor $1$.

  \chapter{Concluzie}

Problemele de optimizare mereu mi s-au părut o parte fascinantă a informaticii.
Exemplele și tehnicile ilustrate în lucrare creează o bază solidă în ceea ce
înseamnă metode alternative pentru calculul cuplajelor în grafuri bipartite.
Desigur, se pot pune întrebări cât de util este să aflăm ce valori diferite
de xor pe muchii putem obține din toate cuplajele perfecte? În formă aceasta
poate părea ceva de nișă fără vreo aplicabilitate practică, însă aceasta stă de
fapt la baza multor altor problemele întâlnite în informatică, cum ar fi:

\begin{itemize}
  \item În teoria jocurilor, soluția faimosului joc NIM este să determini dacă
    suma xor a valorilor din șir este sau nu nenula. Astfel, putem folosi soluția
    problemei pentru a determina dacă există măcar un cuplaj perfect cu suma xor 0.

  \item Cum am văzut și în soluțiile parțiale care foloseau programare dinamică pe
    submultimi, numerele în baza 2 pot reprezenta de fapt o submultime de obiecte,
    iar pentru că nu există transport în cazul adunării xor, obiectele diferite nu
    se vor influența.

  \item Ideea precedentă se poate extinde și pentru alte baze de numeratii, deoarece
    și transformarea Walsh Hadamard se poate \cite{whtbase}. Astfel se poate implementa
    adunarea vectorilor de pe muchiile cuplajului în $\mathbb{Z}_p$. Spre exemplu,
    poate ne interesează să găsim un cuplaj perfect unde pe muchii sunt $K$ tipuri
    de obiecte care trebuie împărțite egal la $p$ oameni. În acest caz lucrăm cu
    numere de $K$ cifre în baza $p$ și dorim iar să obținem suma ``xor'' 0.
\end{itemize}

Și sunt sigur că mai putem găși multe alte exemple care par mult mai familiare,
dar se pot reduce și ele la ``Xor Matching''. De asemenea putem să o extindem în
feluri diferite. Ce proprietăți are operație xor pentru care am putut să rezolvăm
problema? Transformarea Walsh Hadamard se poate modifica pentru a rezolva operații
arbitrare atâta timp cât se găsesc matricele de transformare, cum ar fi or pe biți
sau and pe biți.

Toate problemele s-au rezolvat în contextul matricei Edmonds, însă există o extensie
directă a acesteia pentru grafuri oarecare, matricea Tutte \cite{tutte}. Ce
proprietăți pierdem când facem trecerea și care dintre probleme rămân valabile
odată ce facem această schimbare? În această lucrare am introdus algoritmul de
cuplaj de cost minim pe muchii \ref{mincostmatching}, iar aceasta se știe că este
o problema care admite un algoritm polinomial și pentru grafuri oarecare, dar care
nu este bazat pe flux, dar pe un algoritm similar cu cel ungar în combinație cu
metoda ``Blossom'' inventată de către Jack Edmonds. Acesta a fost considerat mult
timp un algoritm mai complicat de înțeles și implementat iar o metodă bazată pe
algoritmi de algebră liniară și matricea Tutte ar fi o alternativă tentantă, mai
ales în cadrul concursurilor de programare competitivă.

Am pus deja destul de mult accent pe asta, dar consider că programarea competitivă
și cercetarea în informatică au o corelație foarte puternică și mi-aș dori să văd
în viitor mai multe lucrări că aceasta, care se află la marginea dintre cele două.


  \singlespacing
  \addcontentsline{toc}{chapter}{Bibliografie}
  \bibliographystyle{plainurl}
  \bibliography{references}
\end{document}
