\chapter{Context}

\section{Teoria grafurilor}

\begin{defn}
  Un \textbf{graf neorientat} este o pereche ordonata $G(V, E)$, unde:
  \begin{itemize}
    \item{$V$ este multimea de noduri}.
    \item{$E \subseteq \{(x, y)\ |\ (x, y) \in V^2 \land x \neq y\}$ este
      multimea de muchii, care sunt perechi neordonate de noduri}.
  \end{itemize}
\end{defn}

\begin{defn}
  Un \textbf{graf bipartit} este un graf neorientat $G(V, E)$ ale carui noduri pot fi
  partitionate in doua multimi disjuncte $V_{1}$ si $V_{2}$, altfel incat fiecare muchie
  conecteaza un nod din multimea $V_{1}$ cu un nod din multimea $V_{2}$. Atunci,
  se va nota cu $G(V_{1}, V_{2}, E)$.
\end{defn}

\begin{defn}
  \textbf{Matricea Edmonds} $A$ a grafului bipartit $G(U, V, E)$, unde
  $U = \{u_{1}, u_{2}, u_{3}, \ldots, u_{n}\}$,
  $V = \{v_{1}, v_{2}, v_{3}, \ldots, v_{n}\}$ si $x_{i,j}$ indeterminate, se defineste ca:
   \begin{equation}
    A=
    \begin{cases}
      x_{i,j} & \text{daca}\ (u_{i}, v_{j}) \in E \\
      0 & \text{altfel}
    \end{cases}
  \end{equation}
\end{defn}

\begin{thm}
  \label{edmonds}
  Un graf bipartit $G(U, V, E)$ admite cuplaj perfect daca si numai daca
  polinomul determinantului matricei sale Edmonds nu este polinomul nul.
\end{thm}

\begin{clr}
  Numarul de cuplaje perfecte este egal cu numarul monoamelor in polinomul $det(A_{i,j})$.
\end{clr}

\begin{clr}
  Rangul matricei Edmonds este egal marimea cuplajului maximal.
\end{clr}

\pagebreak

\section{Lema Schwartz–Zippel}
\textbf{Lema Schwartz-Zippel} este o metoda utila pentru a verifica intr-un mod
probabilistic daca un polinom in mai multe variabile este polinomul nul.
% TODO: poate discutie despre autori?

\begin{thm}
  Fie $P \in F \lbrack x_{1}, x_{2}, \ldots, x_{n} \rbrack$ un polinom nenul de grad $d \geq 0$
  peste corpul $F$ si $S$ o submultime finita a lui $F$. Alegem
  $r_{1}, r_{2}, \ldots, r_{n}$ uniform aleator si independent din $S$. Atunci:
  \begin{equation}
    \Pr \lbrack P(r_{1}, r_{2}, \ldots, r_{n}) = 0 \rbrack \leq \frac{d}{|S|}
  \end{equation}
\end{thm}

Acest rezultat are sens odata ce ne gandim pe cazul de baza, cand $n=1$ deoarece
un polinom de grad $d$ are cel mult $d$ radacini. Intreaga demonstratie se
bazeaza pe inductie matematica si se poate consulta in \cite{schwartzzippel}.

\begin{thm}
  Lema Schwartz-Zippel se poate folosi in \ref{edmonds} pentru a verifica
  existenta unui cuplaj perfect in timp polinomial.
\end{thm}

\section{Lema de izolare}
\textbf{Lema de izolare} este o metoda de a reduce numarul de solutii ale unei
probleme la una singura, daca aceasta exista. Aceasta se obtine prin adaugarea
unor constrangeri aleatoare in asa fel incat, cu o probabilitate neglijabila,
va exista o singura solutie care satisface constragerile aditionale.

Lema a fost introdusa de Valiant si Vazirani in lucrarea ``NP is as easy as
detecting unique solutions'', publicata in 1986.

\begin{thm}
  Fie $n$ si $m$ doua numere intregi pozitive si $F$ o familie de submultimi a
  lui $\{1, 2, \ldots n\}$. Fie
  $w : \{1, 2, \ldots, n\} \to \{1, 2, \ldots, m\}$ o functie care asociaza
  valori uniform aleatoare si independente din codomeniu. Atunci, cu o
  probabilitate de cel putin $1 - \frac{n}{m}$, va exista o singura multime $S$ in
  $F$ pentru care $\sum_{x\in S}{w(x)}$ este minima.
\end{thm}

\begin{thm}

\end{thm}
