\chapter{Context}

\section{Concepte din algebra liniară}

\begin{defn}
  Un \textbf{monoid} $G = (A, \circ)$ este o structură algebrică formată dintr-o
  mulțime $A$ și o lege de compoziție asociativă și cu element neutru.
\end{defn}

\begin{defn}
  Un \textbf{grup} $G = (A, \circ)$ este o structură algebrică formată dintr-o
  mulțime $A$ și o lege de compoziție care respectă următoarele proprietăți:
  \begin{itemize}
      \item{$\forall\ x_{1}, x_{2} \in G, x_{1} \circ x_{2} \in G$.}
      \item{$\forall\ x_{1}, x_{2}, x_{3} \in G, (x_{1} \circ x_{2}) \circ x_{3} = x_{1} \circ (x_{2} \circ x_{3})$.}
      \item{$\exists\ e \in G \text{ astfel
      încât } \forall\ x \in G\ e \circ x = x \circ e = x$.}
      \item{$\forall\ x_{1} \in G\ \exists\ x_{2} \in G \ x_{1} \circ x_{2} = x_{2} \circ x_{1} = e$.}
  \end{itemize}
  În plus, acesta este abelian dacă
  $\forall\ x_{1}, x_{2} \in G\ x_{1} \circ x_{2} = x_{2} \circ x_{1}$.
\end{defn}

\begin{defn}
  Un \textbf{inel} $I = (A, +, *)$ este o structura algebrică formată dintr-o
  mulțime $A$ și două operații binare definite pe $A \times A$ cu valori tot în
  $A$, care respectă următoarele proprietăți:
  \begin{itemize}
      \item{$(A, +)$ este un grup abelian.}
      \item{$(A, *)$ este un monoid.}
      \item{Distributivitatea lui $*$ față de $+$.}
  \end{itemize}
\end{defn}

\begin{defn}
  Un \textbf{polinom} este o expresie construită dintr-una sau mai multe
  variabile și constante, folosind doar operații de adunare, scădere, înmulțire
  și ridicare la putere constanta pozitivă întreagă.
\end{defn}

\begin{defn}
  O \textbf{matrice} este un tabel dreptunghiular de elemente ale unui inel.
\end{defn}

\subsection{Interpolare Lagrange}

\textbf{Polinomul de interpolare Lagrange} este un polinom $P$ construit în așa
fel încât având $k + 1$ perechi
$(x_{1}, y_{1}), (x_{2}, y_{2}), \ldots, (x_{k+1}, y_{k+1})$ avem că
$P(x_{i}) = y_{i}\ \forall\ 1 \leq i \leq K + 1$. \par
Construcția lui $P$ a fost oferită de Joseph Louis Lagrange în $1795$:
\begin{equation}
  P(x) := \displaystyle\sum\limits_{i_{1}=0}^{k} (y_{i_{1}} * \prod_{1 \leq i_{2} \leq k + 1 \ \land \ i_{1} \neq i_{2}} \frac{x - x_{i_{2}}}{x_{i_{1}} - x_{i_{2}}})
\end{equation}

\section{Eliminare Gaussiană}
\label{gauss}

O metodă de a rezolva sisteme de ecuații liniare este \textbf{Eliminarea Gaussiană}.
Aceasta se mai numește și metoda \textbf{Gauss-Jordan} pentru că este o variație a
metodei lui Gauss, descrisă de către Jordan în anul 1887.

Algoritmul face eliminări secvențiale a tuturor variabilelor până când fiecare ecuație
rămâne cu o singură variabilă. Dacă o privim ca o ecuație $Ax = b$, atunci se poate
vedea că, în cazul în care $A$ este o matrice pătratică, atunci vrem să o transformăm
în matricea identitate și să rezolvăm sistemul pe acest caz trivial.

La primul pas, algoritmul împarte prima linie cu $a_{1, 1}$ și apoi adaugă o combinație
liniară a primei ecuații tuturor celorlalte astfel încât toți ceilalți coeficienți de pe
prima coloană să fie $0$; mai exact pentru rândul $a_{i}, i \geq 2$, vom adaugă primul rând
înmulțit cu $-a_{i}$. Toate aceste operații se execută și pe vectorul coloană $b$. Se
continuă aceleași operații pentru restul liniilor.

Este posibil ca acele valori să fie $0$. De aceea există mai multe euristici, în special gândite
pentru a stabiliza operațiile pe numere reale în virgulă flotantă. Euristica cea mai des folosită
în practică este să se caute valoarea cea mai mare absolută de pe acea coloană și să se interschimbe
cele două rânduri.

Complexitatea algoritmului este de $O(N^{2}M)$ dacă $A$ este de mărime $N \times M$, cu $N \leq M$.
În practică acest algoritm se poate aplica și atunci când se lucrează într-un corp finit.
Un caz des întâlnit este în $\mathbb{F}_{2}$. Pe modelul RAM acest caz se poate implementa chiar în
complexitate $O(\frac{N^{2}M}{w})$, unde $w$ este mărimea unui cuvânt, adică $64$ pentru procesoarele
moderne.

Metoda poate fi folosită și pentru a calcula inversa unei matrice. Tot ce este de făcut, pentru o matrice
$A$ de mărime $N \times N$ este să se execute algoritmul pe matricea $[A\ |\ I_{N}]$ de mărime $N \times 2N$,
iar matricea de pe blocul din dreapta (coloanele  cu indicemai mare ca $N$) va fi chiar $A^{-1}$.

Pentru calculul determinantului, putem executa algoritmul de înainte cu mențiunea că nu vom mai împărți fiecare
linie, dar vom înmulți acele valori de pivot și de fiecare dată când schimbăm două linii, schimbăm semnul
rezultatului. Dacă la un moment dat nu se găsește un pivot, atunci rezultatul este chiar $0$. Dacă se
dorește calculul rangului, atunci aceste este chiar numărul pivoților diferiți de $0$.

\pagebreak

\section{Teoria grafurilor}

\begin{defn}
  Un \textbf{graf neorientat} este o pereche ordonată $G(V, E)$, unde:
  \begin{itemize}
    \item{$V$ este mulțimea de noduri}.
    \item{$E \subseteq \{(x, y)\ |\ (x, y) \in V^2 \land x \neq y\}$ este
      mulțimea de muchii, care sunt perechi neordonate de noduri}.
  \end{itemize}
\end{defn}

\begin{defn}
  Un \textbf{arbore} este un graf neorientat conex aciclic.
\end{defn}

\begin{defn}
  Un \textbf{graf bipartit} este un graf neorientat $G(V, E)$ ale cărui noduri pot fi
  partiționate în două mulțimi disjuncte $V_{1}$ și $V_{2}$, altfel încât fiecare muchie
  conectează un nod din mulțimea $V_{1}$ cu un nod din mulțimea $V_{2}$. Atunci,
  se va nota cu $G(V_{1}, V_{2}, E)$.
\end{defn}

\begin{defn}
  Un \textbf{cuplaj} într-un graf este o mulțime de muchii fără noduri în comun.
\end{defn}

\begin{defn}
  \textbf{Matricea Edmonds} $A$ a grafului bipartit $G(U, V, E)$, unde
  $U = \{u_{1}, u_{2}, u_{3}, \ldots, u_{n}\}$,
  $V = \{v_{1}, v_{2}, v_{3}, \ldots, v_{n}\}$ și $x_{i,j}$ indeterminate, se definește ca:
   \begin{equation}
    A=
    \begin{cases}
      x_{i,j} & \text{dacă}\ (u_{i}, v_{j}) \in E \\
      0 & \text{altfel}
    \end{cases}
  \end{equation}
\end{defn}

\begin{thm}
  \label{edmonds}
  Un graf bipartit $G(U, V, E)$ admite cuplaj perfect dacă și numai dacă
  polinomul determinantului matricei sale Edmonds nu este polinomul nul.
\end{thm}

\begin{clr}
  Numărul de cuplaje perfecte este egal cu numărul monoamelor în polinomul $det(A_{i,j})$.
\end{clr}

\begin{clr}
  Rangul matricei Edmonds este egal cu mărimea cuplajului maximal.
\end{clr}

\pagebreak

\section{Lema Schwartz–Zippel}
\textbf{Lema Schwartz-Zippel} este o metodă utilă pentru a verifica într-un mod
probabilistic dacă un polinom în mai multe variabile este polinomul nul.

\begin{thm}
  Fie $P \in F \lbrack x_{1}, x_{2}, \ldots, x_{n} \rbrack$ un polinom nenul de grad $d \geq 0$
  peste corpul $F$ și $S$ o submultime finită a lui $F$. Alegem
  $r_{1}, r_{2}, \ldots, r_{n}$ uniform aleator și independent din $S$. Atunci:
  \begin{equation}
    \Pr \lbrack P(r_{1}, r_{2}, \ldots, r_{n}) = 0 \rbrack \leq \frac{d}{|S|}
  \end{equation}
\end{thm}

Acest rezultat are sens odată ce ne gândim pe cazul de baza, când $n=1$ deoarece
un polinom de grad $d$ are cel mult $d$ rădăcini. Întreagă demonstrație se
bazează pe inducție matematică și se poate consulta în \cite{schwartzzippel}.

\begin{thm}
  Lema Schwartz-Zippel se poate folosi în \ref{edmonds} pentru a verifica
  existența unui cuplaj perfect în timp polinomial.
\end{thm}

Fie $A$ \textbf{matricea Edmonds} asociată grafului. $\det(A)$ este un polinom
în $n^{2}$ variabile, de grad $n$. Alegem corpul pe care lucrăm că fiind
$\mathbb{F}_{p}$. Determinantul poate fi calculat sub $\mathbb{F}_{p}$ în timp $O(n^{3})$
folosind Eliminare Gaussiana. Chiar mai mult decât atât, acest calcul
poate fi făcut în timp $O(\log^{2} n)$ dacă avem la dispoziție $O(n^{3.5})$
procesoare \cite{paralleldet}. Probabilitatea de eșec a algoritmului este de $\frac{n}{p}$.

\pagebreak

\section{Lema de izolare}
\textbf{Lema de izolare} este o metodă de a reduce numărul de soluții ale unei
probleme la una singură, dacă aceasta există. Această se obține prin adăugarea
unor constrângeri aleatoare în așa fel încât, cu o probabilitate neglijabilă,
va există o singură soluție care satisface constragerile adiționale.

Lema a fost introdusă de Valiant și Vazirani în lucrarea ``NP is as easy as
detecting unique solutions'', publicată în 1986.

\begin{thm}
  \label{isolation}
  Fie $n$ și $m$ două numere întregi pozitive și $F$ o familie de submultimi a
  lui $\{1, 2, \ldots n\}$. Fie
  $w : \{1, 2, \ldots, n\} \to \{1, 2, \ldots, m\}$ o funcție care asociază
  valori uniform aleatoare și independente din codomeniu. Atunci, cu o
  probabilitate de cel puțin $1 - \frac{n}{m}$, va există o singură mulțime $S$ în
  $F$ pentru care $\displaystyle\sum\limits_{x\in S}{w(x)}$ este minimă.
\end{thm}

\pagebreak

\section{Algoritm paralelizabil pentru cuplaj}
\label{Algoritm paralelizabil pentru cuplaj}

În ``Matching is as easy as matrix inversion'' \cite{matchingezmatrix} se
descrie o metodă pentru a căuta cuplajul perfect folosind acest rezultat.
Considerăm $G(U, V, E)$ un graf bipartit. Asociem fiecărei muchii din $E$ un
cost uniform din mulțimea $\{1, 2, \ldots, 2|E|\}$. Conform \ref{isolation}, cu
probabilitate de cel puțin $\frac{1}{2}$ va exista un cuplaj unic de cost minim.
Considerăm $U = \{u_{1}, u_{2}, u_{3}, \ldots, u_{n}\}$,
$V = \{v_{1}, v_{2}, v_{3}, \ldots, v_{n}\}$ și costul aleator ales mai devreme muchiei
$(u_{i}, v_{j})$ că fiind $w_{i,j}$. Fie $A$ matricea Edmonds a acestui
graf, construită în felul următor:
\begin{equation}
A=
\begin{cases}
  2^{w_{i,j}} & \text{dacă}\ (u_{i}, v_{j}) \in E \\
  0 & \text{altfel}
\end{cases}
\end{equation}

Definim costul unui cuplaj ca fiind suma costurilor muchiilor alese în acea mulțime.
\begin{lem}
  Dacă $p$ este costul cuplajului minim și acesta este și unic, atunci $\det(A) \neq 0$, iar
  $2^{p} \ |\ \det(A)$ și $2^{p+1} \not| \ \det(A)$.
\end{lem}

Fie $\text{sgn} : S_{n} \to \{\pm 1\}$, $\text{sgn}(P)$ este $+1$ dacă $P$
este permutare pară, altfel $-1$. $\text{sgn}(P)$ se numește signatura
permutarii $P$.

\begin{equation}
  \label{isolationdet}
  \det(A) = \displaystyle\sum\limits_{P \in S_{n}} (\text{sgn}(P) * \prod_{1 \leq i \leq N} A_{i, P_{i}})
\end{equation}

Pentru fiecare cuplaj din $G$ putem asocia unic o permutare $P$ din $S_{n}$:
dacă muchia $(u_{i}, v_{j})$ este aleasă în cuplaj, atunci $P_{i} \equiv j$.
Pentru o permutare care nu are asociat un cuplaj perfect, aceasta nu va
contribui deloc la valoarea expresiei \ref{isolationdet}. \par

Dacă $P$ este permutarea asociată cuplajului de cost minim, atunci
$\prod_{1 \leq i \leq N} A_{i, P_{i}} = 2^{p}$. Valoarea celorlalte permutări
este fie $0$, fie o putere de $2$ mai mare decât $2^{p}$ (deoarece $P$ este
unică). Așadar, valoarea determinatului o să fie divizibilă cu $2^{p}$ și
mai mult, această va fi cel mai mare divizor de această formă.

\pagebreak

Pentru a determina mulțimea muchiilor din cuplaj, trebuie să mai facem câteva observații.

\begin{lem}
  Fie $A'$ matricea obținută eliminând linia $i$ și coloana $j$. Fie
  $C \def \det(A') * \frac{2^{w_{i,j}}}$. Muchia
  $(u_{i}, v_{j})$ face parte din cuplaj dacă și numai dacă $2^{p} \ | \ C$,
  $2^{p+1} \not| \ C$ și $C \neq 0$.
\end{lem}

În primul rând, valoarea $\det(A') * 2^{w_{i,j}}$ este chiar valoarea
determinantului calculat după formula de la \ref{isolationdet}, cu precizarea că
se vor considera doar permutările $P$ cu $P_{i} = j$. Acest lucru se datorează
faptului că permutările alese în calculul lui $\det(A')$ vor avea cu un element
mai puțin decât cele din $\det(A)$, iar pentru elementul lipsă înmulțim mereu cu
valorea asociată muchiei $(u_{i}, v_{j})$, adică $2^{w_{i,j}}$.
Aplicând apoi observațiile demonstrației pentru \ref{isolationdet} obținem
rezultatul dorit și în cele din urmă, un algoritm pentru determinarea cuplajului
perfect care nu se bazează pe rețele de flux, ci doar pe obiect de algebră
liniară: \par

\vspace{5 mm}

\begin{algorithm}[H]
 \label{cuplajizolare}
 \KwData{Graf bipartit G(U, V, E) care admite cuplaj perfect}
 \KwResult{Mulțimea muchiilor alese într-un posibil cuplaj perfect}
 \Begin{
  determină w\;
  construiește A\;
  $p \longleftarrow \det(A)$\;
  $C \longleftarrow \{\}$\;
  \For{$(u_{i}, v_{j}) \in E$}{
    determină $A'$\;
    \If{$\frac{\det(A') * 2^{w_{i,j}}}{2^{p}} \mod 2 = 1$}{
      $C \longleftarrow C \cup \{(u_{i}, v_{j})\}$\;
    }
  }
  \Return{$C$}\;
 }
 \caption{Cuplaj perfect lema de izolare}
\end{algorithm}

\pagebreak

In continuare vom presupune ca putem efectua operatiile aritmetice de baza in timp $O(1)$.
Atunci algoritmul \ref{cuplajizolare} dureaza $O(n^{5})$, daca folosim
algoritmul lui Gauss pentru calcularea determinantului. Putem sa nu recalculam
valoarea determinantului la fiecare iteratie folosind:
\begin{equation}
  \det(A + uv^{T}) = (1 + v^{T} A^{-1} u) \det(A)
\end{equation}

Cu aceasta proprietatea a determinantului, cunoscuta si ca \textbf{Matrix
  determinant lemma} putem reduce complexitatea la $O(n^{4})$. In plus,
algoritmul se poate paraleliza bine, deoarece determinantul unei matrice
de marime $n \times n$ se poate calcula in timp $O(\log^{2}n)$ pe $O(n^{3.5})$ procesoare.

Unicitatea solutiei, garantata de \textbf{lema de izolare} permite procesoarelor
sa execute si pasul urmator, in care se testeaza o muchie daca face parte din
cuplaj, in paralel, deoarece se indreapta catre aceeasi solutie, cu o
certitudine destul de mare. Asadar si aceasta parte a algoritmului poate fi
calculata in timp polilogaritmic, daca avem la dispozitie un numar polinomial de
procesoare.

In \cite{optimalmatching} se arata ca acest algoritm poate fi rafinat la complexitate
$O(n^{3})$, observandu-se ca o muchie $(u_{i}, v_{j})$ face parte din cuplaj daca si
numai daca $A^{-1}_{i, j} \neq 0$. Mai mult, folosindu-se de aceasta proprietate si
de o schema de a reduce eliminarea Gaussiana (\ref{gauss}) la $O(1)$ inmultiri de matrici,
acest algoritm are complexitate $O(n^{w})$, unde $w < 2.38$, astfel obtinandu-se un
algoritm mai eficient decat cel bazat pe flux maxim, Hopcroft-Karp, care are complexitatea
$O(n^{2.5})$.

\pagebreak

\section{Cuplaj rosu-albastru}
\label{redbluematching}

Sa consideram un graf bipartit $G(U, V, E)$, unde $E_{1}$ este multimea
muchiilor rosii, $E_{2}$ este multimea muchiilor albastre, iar
$E = E_{1} \cup E_{2}$. Dorim sa alfam daca exista un cuplaj perfect
in care se folosesc exact $k$ muchii rosii, stiind ca daca acesta exista, atunci
este si \textbf{unic}. \par

Aceasta problema nu se poate rezolva cu algoritmi de flux maxim, spre deosebire
de celelalte probleme de pana acum. Totusi, avem sanse sa o rezolvam, daca
procedam prin a modifica \textbf{matricea Edmonds} cat sa avem un mod de a ne da seama de
culoarea muchiilor alese in cuplaj:

\begin{equation}
  A=
  \begin{cases}
    y & \text{daca}\ (u_{i}, v_{j}) \in E_{1} \\
    1 & \text{daca}\ (u_{i}, v_{j}) \in E_{2} \\
    0 & \text{altfel}
  \end{cases}
\end{equation}

$\det(A)$ va fi un polinom in $y$ de grad $n$. $G$ va avea un cuplaj perfect cu
$k$ muchii rosii daca coeficientul lui $y^{k}$ va fi nenul. Pentru a determina
acest numar putem sa interpolam polinomul $\det(A)$ folosind interpolarea
Lagrange. Tot ce ramane de facut este sa evaluam polinomul in $n+1$ puncte. \par

Pentru cazul general, cand cuplajul rosu-albastru nu este neaparat unic, putem
folosi una din cele doua leme sa reducem problema la una mai simpla.

\subsection{Folosind Lema Schwartz-Zippel}

Pentru fiecare muchie $(u_{i}, v_{j})$ alegem o valoarea aleatoare $x_{i,j}$ dintr-o
multimea $S$. Construim matricea astfel:

\begin{equation}
  A=
  \begin{cases}
    y * x_{i,j} & \text{daca}\ (u_{i}, v_{j}) \in E_{1} \\
    x_{i,j} & \text{daca}\ (u_{i}, v_{j}) \in E_{2} \\
    0 & \text{altfel}
  \end{cases}
\end{equation}

Ca inainte, vom interpola $\det(A)$ pentru a afla coeficientul lui $y^{k}$.
Acest coeficient trebuie sa fie nenul. Iar acum, cu probabilitate mare, acest
lucru este adevarat, pentru ca daca doua polinoame se ``anuleaza'' unul pe
celalalt in urma calculului determinantului, atunci cu probabilitate de cel mult
$\frac{n}{|S|}$ acestea erau diferite.

\subsection{Folosind Lema de izolare}
Asemanator algoritmului prezentat la \ref{Algoritm paralelizabil pentru cuplaj}
si solutiei precedente cu Lema Schwartz-Zippel, avand valorile $w_{i,j}$ alese
din $\{1, 2, \ldots, 2|E_{1} \cup E_{2}|\}$, vom construi matricea astfel:

\begin{equation}
  A=
  \begin{cases}
    y * 2^{w_{i,j}} & \text{daca}\ (u_{i}, v_{j}) \in E_{1} \\
    2^{w_{i,j}} & \text{daca}\ (u_{i}, v_{j}) \in E_{2} \\
    0 & \text{altfel}
  \end{cases}
\end{equation}

In continuare ramane sa ne uitam la coeficientul lui $y^{k}$ in $\det(A)$. Spre
deosebire de lema \textbf{Schwartz-Zippel}, acesta varianta are posibilitatea sa fie
paralelizabil, deoarece, cu probabilitate de cel putin $\frac{1}{2}$, cuplajul
pe care vrem sa-l gasim este acum unic.

\pagebreak

\section{Transformari Fourier}

\subsection{Transformarea Fourier Discreta}
Consideram $n = 2^{p}$ si urmatorul polinom de grad $n-1$:
\begin{equation}
  A = a_{0}x^{0} + a_{1}x^{1} + \ldots + a_{n-1}x^{n-1}
\end{equation}

Solutia ecuatiei $x^{n} = 1$ are $n$ solutii in multimea numerelor complexe.
Acestea sunt de forma $w_{p} = e^{\frac{2k\pi i}{n}}$ pentru $p$ un numar intreg
intre $0$ si $n-1$. Aceste radacini au multe proprietati utile, iar cea pe care
urmeaza sa o folosim este ca $w_{p} = w_{0}^{p}$. \par
\textbf{Transformarea discreta Fourier} a polinomului $A$ este definita ca fiind
vectorul obtinut prin evaluarea polinomului in radacinile unitatii:

\begin{equation}
  \label{dft}
  \text{DFT}(A) = (A(w_{0}), A(w_{1}), \ldots, A(w_{n-1})) \\
                = (A(w_{0}^{1}), A(w_{0}^{2}), \ldots, A(w_{0}^{n-1}))
\end{equation}

Similar, avand vectorul din partea dreapta a ecuatiei \ref{dft}, \textbf{inversa
transformarii discrete Fourieri} reconstituie coeficientii polinomului $A$:
$(a_{0}, a_{1}, \ldots a_{n-1})$. Vom nota aceasta transformare cu
$\text{InverseDFT}$. \par

O aplicatie utila este inmultirea polinoamelor. Fie $\textbf{dot}$ operatia de
inmultire scalara a doi vectori. Aceasta se bazeaza pe o
proprietatea a transformarii, si anume:
\begin{equation}
  \text{DFT}(A * B) = \text{dot}(\text{DFT}(A), \text{DFT}(B))
\end{equation}

Aceasta proprietate este utila, fiind la baza unui algoritm rapid de inmultire a
polinoamelor, deoarece $\text{DFT}(A)$ si $\text{InverseDFT}(A)$ se pot calcula
in timp $O(n \log n)$.

\subsection{Transformarea Hadamard}
\label{hadamard}

\textbf{Transformarea Hadamard} este o transformare Fourier generalizata,
executand o operatie liniara pe $2^{n}$ numere reale. Aceasta este echivalenta
cu o transformare discreta $n$-dimensionala Fourier de marime
$2 \times 2 \ldots \times 2$. In mod intuitiv, se inmultesc monoame a $n$
variabile, aplicandu-se modulo $x^{2}-1$ pe fiecare variabila. Fie $p_{i_{1}, i_{2}} \in \{0, 1\}$:
\begin{equation}
  A = a_{0} x_{0}^{p_{0, 0}} x_{1}^{p_{0, 1}} \ldots x_{n-1}^{p_{0, n-1}}
  + a_{1} x_{0}^{p_{1, 0}} x_{1}^{p_{1, 1}} \ldots x_{n-1}^{p_{1, n-1}}
  + \ldots
  + a_{n-1} x_{0}^{p_{n-1, 0}} x_{1}^{p_{n-1, 1}} \ldots x_{n-1}^{p_{n-1, n-1}}
\end{equation}

Daca in cazul \textbf{transformarii Fourieri discrete} obtineam
$C_{i+j} = \displaystyle\sum\limits A_{i} B_{j}$, in cazul \textbf{transformarii Hadamard} obtinem
$C_{i \oplus j} = \displaystyle\sum\limits A_{i} B_{j}$, unde $\oplus$ este operatia de sau exclusiv
(xor) pe biti. Intuitiv, de ce se obtine operatie de xor este pentru ca
cele $n$ monoame reprezentau bitii fiecarui numar de la $0$ la $2^{n} - 1$.
Acestora la inmultire li se aplica modulo $x^{2} - 1$, asemanator bitilor in
timpul unei operatii de xor, care este ca o adunare, in urma careia se aplica
modulo $2$, totul efectuat separat pe fiecare bit in parte.

\subsection{Algoritmul Fast Walsh-Hadamard}
\label{fasthadamard}

\textbf{Algoritmul Fast Walsh-Hadamard} este un algoritm eficient pentru a
calcula \textbf{transformarea Hadamard}. O implementare naiva a transformarii
necesita complexitate de timp $O(2^{2n})$, in timp ce algoritmul pe care urmeaza
sa-l prezentam necesita doar $O(n2^{n})$. Algoritmul se bazeaza pe paradigma
divide et impera, impartind o problema de marime $2^{n}$ in doua de marime
$2^{n-1}$, urmarind definitia recursiva a matricei Hadamard:

\begin{equation}
  \label{hadamardmatrix}
  H_{n} =
  \begin{pmatrix}
    H_{n-1} & H_{n-1}\\
    H_{n-1} & -H_{n-1}
  \end{pmatrix}
\end{equation}

Ecuatia \ref{hadamardmatrix} omite factorul de normalizare de
$\frac{1}{\sqrt{2}}$, care poate fi aplicat la final.

\begin{algorithm}[H]
  \DontPrintSemicolon
  \SetKwFunction{FMain}{FWHT}
  \SetKwProg{Fn}{Function}{:}{}
  \Fn{\FMain{$a$, $n$}}{
    \If{$n = -1$}{
      \KwRet\;
    }
    offset := $2^{n - 1}$\;
    \FMain($a[0 \ldots \text{offset}]$, n - 1)\;
    \FMain($a[\text{offset} \ldots 2^{n}]$, n - 1)\;
    \For{i in $0 \ldots 2^{n}$}{
      x := $a_{i}$\;
      y := $a_{i + \text{offset}}$\;
      $a_{i}$ := x + y\;
      $a_{i + \text{offset}}$ := x - y\;
    }
    \KwRet\;
  }
  \;
\end{algorithm}

\begin{algorithm}[H]
  \DontPrintSemicolon
  \SetKwFunction{FMain}{IFWHT}
  \SetKwProg{Fn}{Function}{:}{}
  \Fn{\FMain{$a$, $n$}}{
    \If{$n = -1$}{
      \KwRet\;
    }
    offset := $2^{n - 1}$\;
    \FMain($a[0 \ldots \text{offset}]$, n - 1)\;
    \FMain($a[\text{offset} \ldots 2^{n}]$, n - 1)\;
    \For{i in $0 \ldots 2^{n}$}{
      x := $a_{i}$\;
      y := $a_{i + \text{offset}}$\;
      $a_{i}$ := x - y\;
      $a_{i + \text{offset}}$ := x + y\;
    }
    \KwRet\;
  }
  \;
\end{algorithm}

\pagebreak

\section{Algoritmul lui Coppersmith}
\label{coppersmith}

In cadrul algoritmilor de algebra liniara pentru cuplaj avem des problema de a
calcula determinanti. De obicei, aceste grafuri nu vor fi atat de dense, deci
nici matricea Edmonds (\ref{edmonds}) nu va avea multe elemente nenule.

\textbf{Algoritmul lui Coppersmith} \cite{wiedemann} este un algoritm rapid pentru calcularea
vectorilor nucleu al unei matrice intr-un corp finit. Acesta este in particular
eficient pentru matricele rare, cele care nu au atat de multe elemente nenule. \\

Consideram $A$ o matrice patratica de marime $n \times n$ peste un corp finit $F$, atunci:

\begin{defn}
  Polinomul caracteristic al lui $A$ este definit ca
  $p(\lambda) = \det(\lambda I_{n} - A)$.
\end{defn}

\begin{thm}
  \label{cayleyhamilton}
  \textbf{Cayley-Hamilton} spune ca fiecare matrice patratica intr-un inel
  comutativ isi satisface ecuatia caracteristica, mai exact $p(A) = 0$.
\end{thm}

$A$ are un polinom monic $P$ peste corpul $F$ de grad minim pentru care
$P(A) = 0$. $P$ se numeste polinomul minim al lui $A$. In
cazul matricelor rare, ne asteptam ca acest polinom sa aiba un grad mai mic
decat cel al polinomului caracteristic. $P$ mereu va divide polinomul
caracteristic, iar din \ref{cayleyhamilton}, gradul acestuia nu va fi mai mare
de $n$. Fie $P = a_{0} + a_{1}x + \ldots + a_{n'}x^{n'}$, unde $n' \leq n$.

\begin{lem}
  \label{polminimdet}
  Daca $n' = n$, atunci $P$ este chiar polinomul caracteristic al lui $A$, iar
  determinantul este chiar $(-1)^{n}P(0)$.
\end{lem}

\begin{lem}
  Putem folosi polinomul minimal pentru a calcula determinantul matricei $A$.
\end{lem}

\begin{proof}
  Fie $d$ un vector de $n$ elemente
  alese uniform aleator si independent din $F$, iar
  $D =
  \begin{bmatrix}
    d_{1} & 0 & \dots \\
    0 & d_{2} & \dots \\
    \vdots & \ddots & \\
    0 & \dots & d_{n}
  \end{bmatrix}$. Conform lemei Schwartz-Zippel, matricea $AD$ este singulara cu
  probabilitate de cel mult $\frac{n}{|F|}$. In cazul in care nu este, atunci ne
  asteptam ca polinomul minim sa aiba grad egal chiar cu $n$. In acest caz,
  putem calcula rapid $\det(AD)$ folsind \ref{polminimdet} si apoi sa aflam
  $\det(A) = \frac{\det(AD)}{det(D)}$, iar $\det(D) = \prod_{i=1}^{n} d_{i}$.
\end{proof}

Fie $x'$ un vector de $n$ elemente alese uniform aleator si independent din $F$,
iar $x = Ax'$. Fie $S = [x, Ax, A^{2}x, \ldots]$.
Consideram un alt vector de $n$ elemente, $y$ si urmatorul sir de valori din $F$: $[yx, yAx, yA^{2}x, \ldots ]$.

Stim ca $P(A) = 0$, atunci $\displaystyle\sum\limits_{i=0}^{n'} a_{i} A^{i} = 0$. Obtinem ca si
$\displaystyle\sum\limits_{i=0}^{n'} y (a_{i} (A^{i} x)) = 0$. Intuitiv, stim ca $P(S) = 0$, insa nu
stim sa calculam eficient o secventa $b_{0}, b_{1}, \ldots b_{k}$ astfel incat
$\displaystyle\sum\limits_{i=0}^{k} b_{i} S_{i+o} = 0 \ \forall \ o \geq 0$, pentru ca elementele din $S$ sunt vectori, insa
atunci cand reducem la cazul cu valori, o sansa destul de mare ca secventa $b$
care anihileaza $yS$ sa anihileze si $S$, deci
$\displaystyle\sum\limits_{i=0}^{k} b_{i} M^{i}x = 0$.
Inlocuind cu $x = Ax'$, obtinem ca $M \displaystyle\sum\limits_{i=0}^{k} b_{i} M^{i} x' = 0$, deci
daca $b_{i} M^{i} x'$ este nenul, atunci am obtinem un nucleu al lui $M$.

Pentru a calcula elementele secventei $S$ se folosesc algoritmi pentru gasirea
\textbf{polinomului minim al unei recurente liniare}. Algoritmul
\textbf{Berlekamp Massey} daca $|F|$ este un numar prim (doua corpuri de
aceeasi marime sunt izomorfe), altfel, se factorizeaza
$|F| = p_{1}^{k_{1}} p_{2}^{k_{2}} \ldots p_{t}^{k_{t}}$ si se rezolva $t$
probleme cu $|F_{i}'| = p_{i}^{k_{i}}$ cu algoritmul \textbf{Reeds-Sloane}, iar
rezultatele se combina folosind teorema chineza a resturilor. In ce urmeaza vom
prezenta un alt algoritm echivalent care nu foloseste impartiri, deci nu este
constans la un tip de corp anume \cite{sugiyama}.

\pagebreak

\section{Polinomul minim al unei recurente liniare}

\begin{defn}
  O secventa infinita $a$ cu elemente dintr-un corp $F$ se numeste recurenta
  liniara daca si numai daca exista constantele $c_{1}, c_{2}, \ldots, c_{k}$
  astfel incat
  $a_{t} = a_{t-1}c_{1} + a_{t-2}c_{2} + \ldots + a_{t-k}c_{k} \ \forall \ t > k$.
\end{defn}

\begin{defn}
  Pentru o recurenta liniara $a$ si $c_{0}, c_{1}, \ldots c_{k} \in F$, astfel
  incat
  $c_{0} a_{t} = a_{t-1}c_{1} + a_{t-2}c_{2} + \ldots + a_{t-k}c_{k} \ \forall \ t > k$,
  atunci polinomul $c_{0} x^{k} + c_{1} x^{k-1} + \ldots + c_{k} x^{0}$ se
  numeste anihilator al lui $a$.
\end{defn}

\begin{defn}
  Un ideal al unui inel $R$ generat de $S \subset R$ este
  $\{f = \displaystyle\sum\limits_{i=1}^{r} x_{i}s_{i} \ | \ r \in \mathbb{N}, \ s_{i} \in S, x_{i} \in R\}$.
\end{defn}

\begin{defn}
  Un ideal $<S>$ dintr-un inel $R$ este de tip finit daca admite un sistem finit de generatori:
  $\exists \ s_{1}, s_{2}, \ldots s_{r} \in \ <S>$  astfel incat
  $\forall \ s \in \ <S>, s = \displaystyle\sum\limits_{i=1}^{r} x_{i} s_{i}, \ x_{i} \in R$.
\end{defn}

\begin{defn}
  Un ideal se numeste principal daca admite un sistem de generatori format
  dintr-un singur element.
\end{defn}

\begin{defn}
  Inelul $R$ se numeste principal daca orice ideal este principal.
\end{defn}

\begin{thm}
  $F[x]$ este un inel principal.
\end{thm}

\begin{lem}
  Anihilatoarele lui $a$ formeaza un ideal in $F[x]$.
\end{lem}

\begin{clr}
  Idealul format din anihilatoarele lui $a$ este generat de un polinom monic de
  grad minim. Acesta se numeste polinomul minim al lui $a$.
\end{clr}

\begin{thm}
  Functia generatoare $A(x) = \displaystyle\sum\limits_{n=0}^{\infty} a_{n}x^{n}$ este rationala:
  $A(x) = \frac{P(x)}{Q(x)}$ unde $deg(P) < k$, iar $Q(x) = 1 - \displaystyle\sum\limits_{i=1}^{k} c_{i}x^{i}$.
\end{thm}

\begin{thm}
  Daca stim ca gradul polinomului minim este cel mult $m$, atunci acesta este
  unic determinat de primele $2m$ elemente ale lui $a$; sau altfel spus
  $P(x) \equiv Q(X) A(X) \mod x^{2m}$.
\end{thm}

\noindent \textbf{Identitatea lui Bezout}: daca $g$ este cel mai mare divizor comun al
polinoamelor $a$ si $b$, atunci exista doua polinoame $u$ si $v$ astfel incat
$ua + vb = g$.

In \cite{sugiyama} se arata ca putem executa algoritmul lui Euclid extins pentru
calcularea valorilor $u$ si $v$, alegand $a = A(x) \mod x^{2m}$, iar
$b = x^{2m}$ pana cand $deg(g) < m$.

\pagebreak

\section{Filtrare prin radacini ale unitatii}

\label{rootsofunityfilter}

\textbf{Filtrarea prin radacini ale unitatii} este o tehnica de izolare si
insumare a coeficientilor unui polinom. Aceasta se preteaza, in particular,
atunci cand indexurile sunt periodice modulo $n$.

\begin{thm}
  Daca $p$ este un polinom si $\zeta$ este o radacina a unitatii de ordin
  $n$ ($\zeta^{n} = 1$), atunci $\frac{1}{n} \displaystyle\sum\limits_{i=0}^{n-1} p(\zeta^{i})$
  este suma indexurilor dizibile prin $n$.
\end{thm}

\begin{proof}
  Fie $p = a_{0}x^{0} + a_{1}x^{1} + \ldots + a_{k}x^{k}$. Consideram contributia
  unui index $p$, nedivizibil prin $n$ la suma
  $\displaystyle\sum\limits_{i=0}^{n-1} p(\zeta^{i}) = a_{p}(\zeta^{0} + \zeta^{p} + \zeta^{2p} + \ldots + \zeta^{(n-1)p})$.
  Termenul din partea dreapta este de fapt $\displaystyle\sum\limits_{i=0}^{n-1} \zeta^{i*p} = \frac{\zeta^{p*n} - 1}{\zeta^{p} - 1} = 0$,
  deci se anuleaza. Celelalte indexuri vor avea coeficient $1$ pentru fiecare
  $\zeta^{i}$, deci se vor aduna in total de $n$ ori, motiv pentru care toata suma
  se imparte la $n$.
\end{proof}


\pagebreak

\section{Radacini primitive}
\label{primitiveroot}

In ce urmeaza presupunem ca lucram intr-un corp finit $\mathbb{F}_{p}$, unde $p$ este un numar prim.

\begin{defn}
  Un numar $g$ se numeste radacina primitiva modulo $n$ daca si numai daca pentru orice $a$ coprim cu $p$ exista un alt numar
  $l$ astfel incat $g^{l} = a \mod p$.
\end{defn}

\begin{thm}
  \label{carmichael}
  (Functia Carmichael) Fie $g$ o radacina primitiva, atunci cel mai mic numar $l$ astfel incat $g^{l} = 1$ este $\phi(p)$.
\end{thm}

\begin{thm}
  \label{carmichaelr}
  Reciproca teoremei \ref{carmichael} este adevarata si sta la baza algoritmului pentru aflarea generatorului.
\end{thm}

Un algoritm evident pentru a cauta radacini primitive incearca numerele din $\mathbb{F}_{p}$ pe rand si apoi le calculeaza
puterile pentru a verifica daca sunt diferite. Desi in general radacina primitiva este un numar destul de mic, acest algoritm
nu este cel mai eficient posibil.

Din \ref{carmichaelr} stim ca, pentru un $a$ fixat, daca gasim un $l < \phi(p)$ pentru care $a^{l} = 1$, atunci acesta nu este un
generator. Din \textbf{teorema lui Euler} stim ca orice $a$ coprim cu $p$ satisface $a^{\phi(p)} = 1 \mod p$. Aceste observatii pot
imbunatati algoritmul mentionat anterior, insa, pentru ca noi il folosim pentru $p$ numar prim, atunci $\phi(p) = p - 1$, deci nu
ajuta foarte mult. \textbf{Teorema lui Lagrange} mai spune ca este suficient sa verificam ca $a^{d} \neq 1 \mod p$ pentru toti $d \ |\ \phi(p)$.
Aceasta este deja o imbunatatire foarte mare, deoarece numarul de divizori ai lui $\phi(p)$ este de ordinul $O(p^{\frac{1.066}{\ln \ln p}})$.

Daca $\phi(p) = a_{1}^{b_{1}} a_{2}^{b_{2}} \ldots a_{t}^{b_{t}}$, atunci este de ajuns sa se verifice acei $d = \frac{\phi(p)}{a_{i}}$.

\begin{proof}
  Presupunem ca exista un divizor propriu $d$ al lui $\phi(p)$ astfel incat $g^{d} = 1 \mod p$. Atunci exista un factor prim $a_{i}$
  din factorizarea lui $\phi(p)$ astfel incat $d \ | \ \frac{\phi(p)}{a_{i}}$ si evident $g^{\frac{\phi(p)}{a_{i}}} = 1 \mod p$.
\end{proof}

Complexitatea de timp a acestei solutii este acum $O(g \log^{2} p)$, iar daca presupunem ca ipoteza Riemann ca fiind adevarata, $g = O(\log^{6} p)$.
