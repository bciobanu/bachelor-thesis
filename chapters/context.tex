\chapter{Context}

\section{Teoria grafurilor}

\begin{defn}
  Un \textbf{graf neorientat} este o pereche ordonata $G(V, E)$, unde:
  \begin{itemize}
    \item{$V$ este multimea de noduri}.
    \item{$E \subseteq \{(x, y)\ |\ (x, y) \in V^2 \land x \neq y\}$ este
      multimea de muchii, care sunt perechi neordonate de noduri}.
  \end{itemize}
\end{defn}

\begin{defn}
  Un \textbf{graf bipartit} este un graf neorientat $G(V, E)$ ale carui noduri pot fi
  partitionate in doua multimi disjuncte $V_{1}$ si $V_{2}$, altfel incat fiecare muchie
  conecteaza un nod din multimea $V_{1}$ cu un nod din multimea $V_{2}$. Atunci,
  se va nota cu $G(V_{1}, V_{2}, E)$.
\end{defn}

\begin{defn}
  \textbf{Matricea Edmonds} $A$ a grafului bipartit $G(U, V, E)$, unde
  $U = \{u_{1}, u_{2}, u_{3}, \ldots, u_{n}\}$,
  $V = \{v_{1}, v_{2}, v_{3}, \ldots, v_{n}\}$ si $x_{i,j}$ indeterminate, se defineste ca:
   \begin{equation}
    A=
    \begin{cases}
      x_{i,j} & \text{daca}\ (u_{i}, v_{j}) \in E \\
      0 & \text{altfel}
    \end{cases}
  \end{equation}
\end{defn}

\begin{thm}
  Un graf bipartit $G(U, V, E)$ admite cuplaj perfect daca si numai daca
  polinomul determinantului matricei sale Edmonds nu este polinomul nul.
\end{thm}

\begin{clr}
  Numarul de cuplaje perfecte este egal cu numarul monoamelor in polinomul $det(A_{i,j})$.
\end{clr}

\begin{clr}
  Rangul matricei Edmonds este egal marimea cuplajului maximal.
\end{clr}
