\thispagestyle{empty}
\begin{center}
    \Large
    \textbf{Algoritmi pentru cuplaje \\ în grafuri bipartite}

    \vspace{0.4cm}
    {Ciobanu Bogdan}

    \vspace{0.9cm}
    \textbf{Rezumat}

\end{center}
\onehalfspacing
\paragraph{}
În această lucrare ne propunem să prezentăm metode alternative pentru calcule
legate de cuplaje în grafuri bipartite. Plecăm de la algoritmi care folosesc
metode algebrice în detrimentul celor uzuale, bazate pe rețele de flux, și
facem o analiză asupra algoritmilor eficienți pentru operații de algebră
liniară pentru a înțelege cum îi putem folosi pentru a calcula cuplaje.
Folosim metode randomizate și demonstrăm probabilitațiile lor de succes.
Ne folosim de structura lor pentru a rezolva probleme care nu se pot rezolva
cu flux, iar pentru cele care se pot rezolva și cu flux, arătăm cum putem să
obținem complexități cel puțin la fel de bune, cu algoritmi care apoi se pot
paraleliza mai ușor, iar în practică sunt mai pretabili hardware-ului modern.

\begin{center}
    \Large
    \vspace{0.9cm}
    \textbf{Abstract}
\end{center}
\onehalfspacing
\paragraph{}
In this paper we will explain alternative methods for computing matchings in
bipartite graphs. We are starting from algorithms which use algebraic tehniques
as opposed to the usual choice, network flows, and we will examine how efficient
linear algebra algorithms can be used in matchings computation. We are using
randomized methods and we are proving their probability of success. We use their
structure to solve problems that could have never been solved with network flows,
but other problems as well, to show that we can achieve a theoretical time
complexity that is as good as their counterpart, but benefiting from the
efficiency of algebraic operations on modern hardware.
