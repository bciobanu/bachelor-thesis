\chapter{Introducere}

Cuplajul este o problemă fundamentală în teoria grafurilor. Aceasta are
aplicații vaste în informatica teoretică, fiind la baza unor probleme precum
izomorfismul de subarbori, acoperirea unui graf orientat aciclic cu număr minim
de cai disjuncte și detectarea obiectelor aflate în mișcare, dar și în chimie
pentru structura Kekule și indexul Hosoya. \par

De-alungul timpului s-au dezvoltat mulți algoritmi eficienți, din punct de
vedere al complexității de timp, pentru rezolvarea problemelor de cuplaj pe
diferite tipuri de grafuri. \par

În această lucrare ne vom concentra pe grafurile bipartite. În particular pentru
aceștia, cuplajul maximal, o problemă fundamentală de optimizare, se poate
calcula eficient folosind algoritmi de flux maxim. Acest lucru este totuși și o
piedică pentru arhitectură calculatoarelor moderne, fiind grei de paralelizat și
optimizat pentru accesul memoriei cache. De asemenea, pentru aceștia nu se
cunosc extensii în timp polinomial pentru aflarea altor rezultate teoretice
utile, cum ar fi numărul de cuplaje perfecte. Acest lucru a avut totuși să se
schimbe în 1995, când R. Motwani și P. Raghavan au arătat cum se poate folosi
matricea Edmonds, numită după Jack Edmonds care a avut multe contribuții în
domeniul optimizarilor, și algoritmi de algebră liniară pentru a afla mărimea
cuplajului maximal, în lucrarea ``Randomized algorithms''\cite{randomizedalgorithms}. \par

Algoritmii de algebră liniară au avantajul că admit implementări eficiente pe
procesoarele moderne, având un potențial mult mai mare pentru paralelism și
concurență. Observațiile pornind de la matricea lui Edmonds ne vor ajută să rezolvăm
probleme care nu au mai fost rezolvate până acum în timp polinomial. \par

Lucrarea este împărțită în două părți: contextul și problemele. Contextul este o
baza de informații folosite pentru a rezolve problemele. Acesta este compus în
mare parte de noțiuni de teoria grafurilor și algebră, cu accentul pus pe cea din
urmă. Apoi, prezentăm o colecție de problemele, unde fiecare a fost aleasă pentru
a ilustra:
\begin{itemize}
    \item fie cum algoritmii de cuplaj bazați pe algebră liniară o pot rezolva în
    timp polinomial, în timp ce algoritmii de flux nu pot îmbunătăți complexitatea
    exponențială.
    \item fie cum o problema care în aparență nu s-ar preta algoritmilor de cuplaj
    bazați pe algebră liniară, dar se poate rezolva cu flux, se poate de fapt rezolva
    și cu prima metodă având complexitate de timp cel puțin la fel de bună.
\end{itemize}

Problemele sunt alese fie din folclor, fie propuse la concursuri de
programare competitivă. Dintre acestea, problema ``Xor Matching'' a fost propusă
chiar de mine în 2018 și a fost inspirația pentru a scrie această lucrare.
