\chapter{Introducere}

Cuplajul este o problema fundamentala in teoria grafurilor. Aceasta are
aplicatii vaste in informatica teoretica, fiind la baza unor probleme precum
izomorfismul de subarbori, acoperirea unui graf orientat aciclic cu numar minim
de cai disjuncte si detectarea obiectelor aflate in miscare, dar si in chimie
pentru structura Kekule si indexul Hosoya. \par % TODO scrie explicatie aici

De-alungul timpului s-au dezvoltat multi algoritmi eficienti, din punct de
vedere al complexitatii de timp, pentru rezolvarea problemelor de cuplaj pe
diferite tipuri de grafuri. \par

In aceasta lucrare ne vom concentra pe grafurile bipartite. In particular pentru
acestia, cuplajul maximal, o problema fundamentala de optimizare, se poate
calcula eficient folosind algoritmi de flux maxim. Acest lucru este totusi si o
piedica pentru arhitectura calculatoarelor moderne, fiind greu de paralelizat si
optimizat pentru accesul memoriei cache. De asemenea, pentru acestia nu se
cunosc extensii in timp polinomial pentru aflarea altor rezultate teoretice
utile, cum ar fi numarul de cuplaje perfecte. Acest lucru a avut totusi sa se
schimbe in 1995, cand R. Motwani si P. Raghavan au aratat cum se poate folosi
matricea Edmonds, numita dupa Jack Edmonds care a avut multe contributii in
domeniul optimizarilor, si algoritmi de algebra liniara pentru a afla marimea
cuplajului maximal, in ``Randomized algorithms'' \cite{randomizedalgorithms}. \par
