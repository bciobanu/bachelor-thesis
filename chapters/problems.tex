\chapter{Probleme}

\section{Divisible Matching}

\begin{tabular}{l@{\extracolsep{1cm}}l}
  Autor: & Konstantin Khadaev\\
  Concurs: & CS Academy Round \#67\\
  Limita de timp: & 4\ s\\
  Limita de memorie: & 256\ MB\\
\end{tabular}

\hspace{1cm}

\noindent \textbf{Enunt.} Se da un graf bipartit $G(U \cup V, E)$ cu $1 \leq |U| = |V| \leq 100$,
un numar intreg $2 \leq K \leq 100$ si o functie pentru a determina valoarea unei muchii
$f : E \to \{0, 1, \ldots, K-1\}$. Sa se determine daca exista un cuplaj perfect $M$,
astfel incat $K \ | \ \sum_{e \in M} f(e)$.

\hspace{1cm}

\noindent \textbf{Solutie.} Fara a pierde din generalitate, presupunem ca $|U| = |V| = N$.
O solutie evidenta in complexitate $O(N!)$ construieste si verifica toate permutarile din $S_{N}$.
Chiar daca le procesam in ordine aleatoare, numarul asteptat de pasi este tot de ordinul $N!$. \\
Urmatoarea solutie pe care o putem aborda este sa optimizam solutia de backtracking mentionata anterior
cu metoda programarii dinamice. Construim tabela $D_{i,\text{rem}, S}$ ca fiind o valoare booleana daca poti
cupla nodurile $\{u_{1}, u_{2}, \ldots, u_{i}\}$ din $U$ cu nodurile $S \subseteq V$ astfel incat restul sumei
valorilor muchiilor alese de pana acum, modulo $K$ este rem. Recurenta cu care se poate construi este:

\begin{equation}
  D_{i, \text{rem}, S} = \bigvee_{j \in S \wedge e_{i, j} = (u_{i}, v_{j})} D_{i-1, \text{rem} - f(e_{i, j}), S - \{\j\}}
\end{equation}

\noindent Din pacate solutie ramane exponentiala, complexitatea acesteia fiind $O(KN^{2}2^{N})$.

\pagebreak

In ce urmeaza ne putem gandi sa rezolvam o subproblema mai simpla, mai exact cea in care $K=2$.
Ne intereseaza un cuplaj cu numar par de muchii de $1$. Daca privim muchiile cu $1$ ca fiind
rosii, iar celelalte ca fiind albastre, atunci obtinem problema \textbf{cuplajului rosu-albastru} (\ref{redbluematching}),
in care ne intereseaza doar ca numarul de muchii rosii sa fie par. In ce urmeaza ne vom concentra pe varianta
in care folosim lema Schwartz-Zippel. Avand \textbf{matricea Edmonds} $A$ construita special pentru acest caz,
$\det(A)$ este un polinom monic in variabila $y$. Atunci cand voiam sa aflam daca exista un cuplaj cu numar
fix de muchii rosii, interpolam $\det(A)$ si verificam daca $y^{k}$ are un coeficient nenul. In cazul de fata
ne intereseaza daca exista macar un termen $y^{k}$ cu coeficient nenul, iar $k$ par. O prima varianta este sa
interpolam polinomul pentru fiecare numar par, insa putem face mai eficient. O metoda sa aflam suma coeficientilor
pari ai unui polinom $p$ este sa evaluam $\frac{p(1) + p(-1)}{2}$. Desigur, in cazul nostru, acest calcul se
efectueaza tot intr-un corp finit si putem sa obtinem iar $0$ pentru ca s-au anulat coeficientii adunatii.
Din fericire, putem aplica iar lema Schwartz-Zippel sa obtinem ca acest fapt este, in continuare, indeajuns
de improbabil.

Pentru a continua sa rezolvam problema pentru orice $K$, amintim ca insumarea coeficientilor divizibil cu o anumita valoare
se poate face folosind \textbf{filtrarea prin radacini ale unitatii} (\ref{rootsofunityfilter}). Problema este insa
sa gasim un corp potrivit care are radacini ale unitatii de ordinul $K$. Daca nu gasim in timp util, mai putem sa extindem
corpul in asa fel incat sa introducem $\zeta \neq 1$ pentru care $\zeta^{K} = 1$. Insa, daca abordam asa, operatiile
aritmetice vor dura $O(K^{2})$ (sau macar $O(K \log K)$ daca folosim algoritmi de aritmetica "state of the art"),
in loc de $O(1)$, unde tinteam. Corpul pe care il cautam va fi $\mathbb{F}_{p}$, unde $p$ este un numar prim pentru care
$p \mod k = 1$. Pe modelul RAM operatiile aritmetice in $\mathbb{F}_{p}$ dureaza $O(1)$. Presupunem ca generatorul acestui
corp este $g$, pe care il putem gasi cu \textbf{algoritmul pentru radacini primitive} \ref{primitiveroot}. Stim ca
$\phi(p) = p - 1$, deci $g^{p - 1} = 1 \mod p$, iar $k\ |\ p - 1$, deci $\zeta = g^{\frac{p-1}{k}}$.

Complexitatea acestei solutii este acum $O(K W(N))$, unde $W(N)$ este timpul necesar calculului de determinant si este
de ajuns sa se incadreze in limitele date.

\section{Xor Matching}

\begin{tabular}{l@{\extracolsep{1cm}}l}
  Autor: & Bogdan Ciobanu\\
  Concurs: & Codechef September Challenge 2018 Division 1\\
  Limita de timp: & 2\ s\\
  Limita de memorie: & 2\ GB\\
\end{tabular}

\hspace{1cm}

\noindent \textbf{Enunt.} Se da o matrice $A$ de marime $N \times N \ (1 \leq N \leq 60)$ cu valori intregi in celule
din multimea $\{0, 1, \ldots, 1023\}$. Se considera toate permutarile de coloana ale matricei $A$ si se calculeaza
$A_{1, 1} \oplus A_{2, 2} \oplus \ldots \oplus A_{N, N}$, unde $\oplus$ este operatie de xor pe biti.
Sa se determine ce valori se pot obtine.

\hspace{1cm}

\noindent \textbf{Solutie.} O prima solutie este sa consideram toate cele $N!$ permutari ale coloanelor si sa se calculeze
valorile.

O imbunatatire a solutiei precedente este sa aplicam programare dinamica. Mai exact, vom contrui recurenta $D_{i, \text{sum}, S}$
ca fiind o valoare booleana care indica daca putem permuta pentru primele $i$ linii coloanele din multimea $S$ astfel incat sa
suma xor pana acum sa fie egala cu $\text{sum}$. Recurenta este urmatoarea:

\begin{equation}
  D_{i, \text{sum}, S} = \bigvee_{j \in S} D_{i, \text{sum} \oplus A_{i, j}, S - \{j\}}
\end{equation}

Solutia ramane apoi in acele valori adevarate $\text{sum}$ din $D_{N, \text{sum}, \{1, 2, 3, \ldots, N\}}$. Complexitatea acestei solutii
este $O(N^{2}2^{N}\max(A))$, care din pacate nu este destul de rapida pentru restrictiile date.

La prima vedere nu este evidenta relatia pe care problema o are cu cuplajul in grafuri bipartite. Insa daca privim liniile matricei ca
o secventa de noduri $u_{1}, u_{2}, \ldots, u_{N}$, iar coloanele $v_{1}, v_{2}, \ldots, v_{N}$, atunci $A$ este matricea de adiacenta a
unui graf bipartit in care exista o muchie intre oricare doua noduri $u_{i}, v_{j}, 1 \leq i, j \leq N$ de valoare $A_{i, j}$.

Vom incerca sa rezolvam o problema mai simpla acum, mai exact cazul cand $\max(A) = 1$. In acest caz, putem obtine suma xor $0$ daca exista
un cuplaj cu numar par de muchii de valoare $1$, iar suma $1$ daca exista un cuplaj cu numar impar de muchii de valoare $1$. Aceasta subproblema
aduce aminte la problema cuplajului rosu-albastru (\ref{redbluematching}). Ca pana acum, vom considera varianta cu lema Schwartz-Zippel pentru
simplitatea operatiilor aritmetice. Avand \textbf{matricea Edmonds} $A$ construita special pentru acest caz,
$\det(A)$ este un polinom monic in variabila $y$. Atunci cand voiam sa aflam daca exista un cuplaj cu numar
fix de muchii rosii, interpolam $\det(A)$ si verificam daca $y^{k}$ are un coeficient nenul. In cazul de fata
ne intereseaza daca exista macar un termen $y^{k}$ cu coeficient nenul, iar $k$ e par pentru cazul cu suma
xor $0$, iar impar pentru cazul cu suma xor $1$.
